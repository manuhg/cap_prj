\documentclass [11pt, proquest] {uwthesis}[03/03/15] %[2021/06/15]
\usepackage{graphicx}
\usepackage[sort&compress,numbers]{natbib}
\usepackage{multirow}
\usepackage{caption}
\usepackage{subcaption}
\usepackage[breaklinks,hidelinks]{hyperref}
\usepackage{amsmath,amssymb,amsfonts,amsthm,epsfig,epstopdf,titling,url}
\usepackage{enumerate}
\usepackage{algorithm,algorithmic}
\usepackage{{booktabs}}
% \usepackage{tikz}
% \usepackage{tabularx}
% \usepackage{booktabs} % for better table lines
% \usepackage{array} 
% \renewcommand\tabularxcolumn[1]{>{\raggedright\arraybackslash}p{#1}}
\usepackage{enumitem}


\usepackage[numbers]{natbib}
\newenvironment{packed_enum}{
\begin{enumerate}[a)]
  \setlength{\itemsep}{1pt}
  \setlength{\parskip}{0pt}
  \setlength{\parsep}{0pt}
}{\end{enumerate}}

\newenvironment{packed_enum1}{
\begin{enumerate}[1.]
  \setlength{\itemsep}{1pt}
  \setlength{\parskip}{0pt}
  \setlength{\parsep}{0pt}
}{\end{enumerate}}

\newenvironment{packed_enum_par}{
\begin{enumerate}[1)]
  \setlength{\itemsep}{1pt}
  \setlength{\parskip}{0pt}
  \setlength{\parsep}{0pt}
}{\end{enumerate}}

\newenvironment{packed_e}{
\begin{enumerate}
  \setlength{\itemsep}{1pt}
  \setlength{\parskip}{0pt}
  \setlength{\parsep}{0pt}
}{\end{enumerate}}

\newenvironment{packed_itemize}{
\begin{itemize}
  \setlength{\itemsep}{1pt}
  \setlength{\parskip}{0pt}
  \setlength{\parsep}{0pt}
}{\end{itemize}}

% ==========   Local defs and mods

% ----------- definitions format
\theoremstyle{definition}
\newtheorem{defn}{Definition}
\newtheorem{conj}{Conjecture}[section]
\newtheorem{exmp}{Example}[section]



% ================================
% DOCUMENT
% ================================
\begin{document}

\prelimpages
% \raggedbottom


% ---------------------------

\Title{Project TLDR: Standalone desktop application for question answering and summarization using resource-efficient LLMs}

\Author{Manu Hegde}
\Year{2025}
\Program{Computer Science and Software Engineering}

\Chair{Erika Parsons}{}{School of Science, Technology, Engineering \& Mathematics}
\Signature{Michael Stiber}
\Signature{Shane Steinert-Threlkeld}

\copyrightpage

\titlepage  


\setcounter{page}{-1}
\abstract{

This project presents the design and development of a standalone desktop application for offline question answering and summarization over a user-provided document corpus, using resource-efficient large language models (LLMs). Targeted for Apple’s M1/M2 hardware, the application leverages on-device computation via the Apple Neural Engine (ANE) and Metal shaders, exploring the use of the NPU beyond traditional CoreML applications. The application addresses key concerns around data privacy, resource efficiency, and accessibility. Unlike cloud-based services that require constant internet access and raise privacy risks, this application offers a secure, local alternative optimized for researchers and students. It features a graphical interface and supports retrieval-augmented generation (RAG) over the user's corpus, all while utilizing only a fraction of system resources to support seamless multitasking. Evaluation is conducted using both functional metrics (e.g., BERTScore against ChatGPT outputs) and non-functional metrics (e.g., memory and CPU usage). The result is a practical, efficient application that enables interaction with large academic corpora while preserving system responsiveness and data confidentiality.


}

%
% ----- contents & etc.
%
\tableofcontents
\listoffigures
\listoftables  

%
% ----- acknowledgments
%
\acknowledgments{% \vskip2pc
  % {\narrower\noindent
I would like to express my gratitude to Prof. Erika Parsons for all the valuable guidance and help during this work. Furthermore, I sincerely thank Prof. Steinert-Threlkeld and Prof. Stiber for accepting my request to be on the committee for this thesis and for providing precise feedback.
  % \par}
}
%
% end of the preliminary pages
 
 
 
%
% ==========      Text pages
%
\textpages
% ============================================================
%
%                   CHAPTER 1: INTRODUCTION
%
% =============================================================

\chapter {Introduction}
\label{ch:Introduction}

%----------------------
\section{Background and Motivation}
\label{sec:BackgroundAndMotivation}
%----------------------

The field of Natural Language Processing (NLP) has undergone a significant transformation with the advent of Large Language Models (LLMs), which are capable of performing complex language understanding and generation tasks. Groundbreaking works such as the Transformer architecture \cite{vaswani2017attention}, BERT \cite{devlin2018bert}, and GPT-family models \cite{brown2020language, openai2023gpt4} have paved the way for highly capable models that support applications such as summarization \cite{liu2019text}, question answering \cite{izacard2021leveraging}, and document understanding \cite{beltagy2020longformer}. These advances have been further systematized in the concept of foundation models \cite{bommasani2021opportunities}, which emphasize the broad applicability and adaptability of pre-trained LLMs.

Despite their success, most widely used LLM applications operate via cloud-based services, which introduce significant limitations when it comes to privacy, data security, and control over computational resources. This is particularly concerning in academic contexts, where students and researchers often deal with sensitive or proprietary content. Recent studies have raised awareness of the risks associated with exposing private data to generative models, including membership inference \cite{mattern2023membership} and data extraction attacks \cite{nasr2023extraction}. Moreover, surveys indicate increasing usage of LLMs in research and education, highlighting both the demand for such tools and the concerns around data governance \cite{deschenes2024survey, hosseini2023chatgpt}.

Simultaneously, the hardware landscape has evolved to enable local deployment of such models. Apple's M1 and M2 chipsets integrate high-performance CPUs, GPUs, and a dedicated Neural Processing Unit (NPU) through the Apple Neural Engine (ANE). These architectures offer a promising platform for efficient, on-device inference of LLMs, provided the models are adapted appropriately to operate under limited memory and compute budgets.

This convergence of high-capability models, growing privacy concerns, and increasingly powerful consumer hardware forms the backdrop for \textit{Project TLDR}—a standalone desktop application for summarization and question answering over a user-specified corpus. The tool is designed to run entirely offline, preserving user privacy while leveraging optimized LLM inference. The project makes use of modern techniques such as quantization \cite{jacob2017quantization} and low-rank adaptation (LoRA) \cite{hu2021lora} to reduce computational overhead and improve deployment feasibility on M1/M2 hardware. Additionally, the use of Retrieval-Augmented Generation (RAG) \cite{lewis2020rag} ensures that answers and summaries are grounded in user-provided text, enhancing both contextual relevance and factual consistency.

In essence, this project is motivated by the goal of empowering academic users with a practical, secure, and efficient means of engaging with large volumes of textual data. By tying together advances in NLP, secure computing practices, and consumer-grade hardware acceleration, Project TLDR aims to demonstrate that high-quality language understanding can be brought directly to the user's device—without compromise.
%----------------------
\section{Our Contributions}
\label{sec:OurContributions}
%----------------------

In this project, we present \textit{Project TLDR}, a privacy-preserving, offline, and resource-efficient desktop application that enables users to perform Question Answering (QA) and Summarization over personal document repositories. Designed primarily for MacOS systems powered by Apple’s M1 and M2 architectures, the application aims to support academic and research workflows where confidentiality, simplicity, and efficiency are paramount.

Our key contributions are as follows:

\begin{itemize}
    
    \item \textbf{Novel Utilization of Apple Neural Engine (ANE)}: A significant technical contribution of this project is our investigation into utilizing Apple’s underused Neural Processing Unit (ANE), capable of up to 11 TOPS in INT8 precision \cite{apple2024ane}. While current LLM deployment frameworks such as LLaMA.cpp \cite{llamacpp2023} or Ollama\cite{ollama2023} do not harness this co-processor, we demonstrate and document methods to tap into the ANE for local inference acceleration. We build on the NPU API reverse-engineering work by tinygrad \cite{tinygrad2023ane} and leverage the learnings to open a new direction for efficient LLM deployment on Apple silicon devices. We hence leverage the NPU outside of traditional CoreML model deployment paradigm and demonstrate how it can be used for various use cases.

    \item \textbf{Retrieval-Augmented Generation (RAG) Architecture}: We implement a lightweight yet effective RAG pipeline \cite{lewis2020rag} for performing QA and summarization tasks over local collections of documents. This enables the application to provide context-grounded, source-aware responses from user-specified corpora while leveraging limited compute resources.

    \item \textbf{Efficient On-Device Inference Using Quantized LLMs}: We leverage quantized transformer models \cite{jacob2017quantization}, reducing memory and compute demands without compromising output quality. Instead of multi-gigabyte model downloads (as required by tools like Ollama \cite{ollama2023} or LLaMA.cpp \cite{llamacpp2023}), we use compact models (50–500MB) that support practical usage scenarios with minimal setup, enhancing portability and usability for non-technical users.

    \item \textbf{User-Friendly and Ready-to-Use Design}: Unlike tools such as Ollama \cite{ollama2023} and LLaMAFile\cite{llamafile2023}, which require technical familiarity and understanding the nuances of various models, our application provides a clean graphical interface with ready-to-use capabilities tailored to common academic needs—eliminating the steep learning curve and reducing operational friction.
    
    \item \textbf{Privacy-Preserving Document Analysis}: By running entirely on-device, our application mitigates the risks associated with uploading sensitive or proprietary documents to third-party services (e.g., ChatGPT, Claude, Gemini), which have raised concerns over data leakage \cite{mattern2023membership,nasr2023extraction}. Users can securely summarize, query, and rephrase information without network access or cloud APIs.

\end{itemize}

Through this suite of contributions, \textit{Project TLDR} demonstrates that meaningful and secure LLM-powered applications can be brought directly to end-users without reliance on cloud services or specialized technical knowledge, thereby filling a critical gap in the current LLM applications ecosystem.


%----------------------
\section{Scope}
\label{sec:Scope}
%----------------------
The proposed solution focuses on the simple case of a fluid flow interacting with a stationary shape in a 2-dimensional environment and replicates the fluid's behavior using a DL model as accurately as possible while improving the execution time compared to a traditional simulation. 

Because the Navier-Stokes equations used in fluid dynamics are so complex and chaotic (See Section~\ref{ch:TheoreticalBackground}), finding an analytical solution for some problems is impossible. This is why numerical techniques are used to approximate the solutions. Research and industry rely on approximated results to perform their experiments and designs. This means that even when the model solution results are not so precise but provide a fast approximation of the data, it still has value since it is a tool to quickly iterate initial designs that can later be validated with more accurate but slow and resource-demanding methods.

%----------------------
\section{Paper overview}
\label{sec:PaperOverview}
%----------------------
This paper is organized as follows: Chapter~\ref{ch:TheoreticalBackground} explains the main concepts for this work related to Computational Fluid Dynamics and Deep Learning. Chapter~\ref{ch:RelatedWork} presents related work and the current state of DL research for CFD and discusses previous related research relevant to this study. Chapter~\ref{ch:Methods} explains all the methods involved in developing this research and the solution, including the data collection, the DL model architecture and training, and the evaluation techniques. Chapter~\ref{ch:Results} shows the results with its analysis and discussion. Finally, Chapter~\ref{ch:Conclusion} presents the conclusions of this research, its limitations, and future work based on the results obtained.



% ============================================================
%
%                   CHAPTER 2: THEORETICAL BACKGROUND
%
% =============================================================

\chapter {Theoretical Background}
\label{ch:TheoreticalBackground}

This chapter includes the theoretical background for this work discussing related concepts in Computational Fluid Dynamics and Deep Learning.

%----------------------
\section{Fluid dynamics and Navier-Stokes equations}
\label{sec:FluidDynamicsAndNavier-StokesEquations} 
%----------------------

Fluid dynamics is the branch of physics that studies the motion of fluids, both liquids and gases and their interactions with solid boundaries and other fluids. It encompasses a wide range of phenomena, from the flow of water in rivers and the atmosphere's motion to blood circulation in organisms and the behavior of fluids in engineering systems.

The Navier-Stokes equations (See Equations~\ref{eq:navier-stokes}) are fundamental partial differential equations governing the motion of viscous fluids. They describe how fluid velocity, pressure, density, and viscosity evolve over time in response to external forces. The equations are derived from Newton's second law of motion, conservation of mass, and conservation of momentum principles.

The equations consist of two main components: 1) the continuity equation, which represents the conservation of mass, and 2) the conservation of momentum equation, derived from Newton's second law, which describes how the velocity of the fluid changes in response to external forces and internal forces (pressure and viscosity). This is defined in Equation~\ref{eq:navier-stokes} with $t$ time, $\rho$ density, $u$ velocity, $p$ pressure, $\mu$ viscosity, and $F$ external forces.

\begin{equation}
    \begin{aligned}
        \nabla \cdot u &= 0 \\
        \rho(\frac{\partial u}{\partial t} + (u\cdot\nabla)u) &= -\nabla p + \mu \nabla^2 u + \rho F
    \end{aligned}
    \label{eq:navier-stokes}
\end{equation}

The Navier-Stokes equations are nonlinear, leading to complex and sometimes chaotic behavior, such as turbulence. Despite their simplicity, solving these equations analytically for many practical problems is often impossible, leading to the widespread use of numerical methods.

The Navier-Stokes equations have vast applications in various fields, including engineering, meteorology, oceanography. They support the design of aircraft, ships, and vehicles, studying weather patterns, and understanding fluid flow in pipes and channels. However, many fundamental aspects of these equations, including turbulence, remain unsolved problems in mathematics and physics.

%----------------------
\section{Turbulent flow and Reynolds number}
\label{sec:TurbulentFlowAndReynoldsNumber} 
%----------------------

Turbulent flow is characterized by chaotic and unpredictable fluid motion, with irregular fluctuations in velocity, pressure, and flow patterns. It occurs when inertial forces dominate over viscous forces, leading to mixing and eddy formation. The Reynolds number (Re) is a dimensionless parameter that characterizes the flow regime of a fluid. When Re is high, it is a turbulent flow, indicated by the visual appearance of swirling patterns (or eddies), while low Re values indicate a laminar flow type, recognized by a "smooth" flow of the fluid. Thus, turbulent flow is directly correlated to Reynolds numbers, with higher Re values corresponding to more turbulent behavior and lower Re values indicating laminar flow. Equation~\ref{eq:reynolds-numbers} defines Re, with $\rho$ fluid density, L length scale, U velocity, and $\mu$ viscosity.

\begin{equation}
    Re = \frac{\rho LU}{\mu}
    \label{eq:reynolds-numbers}
\end{equation}

%----------------------
\section{Direct Numerical Simulations, RANS and LES}
\label{sec:DirectNumericalSimulationsRANSAndLES} 
%----------------------

In computational fluid dynamics (CFD), Direct Numerical Simulation (DNS), Reynolds-Averaged Navier-Stokes (RANS), and Large Eddy Simulation (LES) are three common approaches for simulating fluid flows, each with its own set of advantages and limitations.

DNS directly solves the Navier-Stokes equations without any modeling assumptions, providing detailed information on all scales of motion in the flow. However, DNS requires high computational resources and is typically only feasible for relatively low Reynolds number flows due to its high computational cost scaling with the cube of the Reynolds number.

RANS averages the Navier-Stokes equations over time to obtain mean flow quantities and then models the effects of turbulent fluctuations using empirical turbulence models. RANS is computationally less expensive than DNS and is suitable for a wide range of engineering applications. However, RANS relies on turbulence models that introduce modeling errors and uncertainties, particularly for complex flows.

LES resolves large-scale turbulent structures explicitly while modeling the effects of smaller-scale turbulence. It strikes a balance between the accuracy of DNS and the computational cost of RANS, making it suitable for simulating moderately high Reynolds number flows. LES captures the essential features of turbulence while reducing modeling errors compared to RANS.

\begin{figure}[H]
    \centering
    \includegraphics[width=0.4\linewidth]{images/dns_les_rans.png}
    \caption{Comparison between DNS, LES, and RANS modeling}
    \label{fig:dns_les_rans}
\end{figure}

The choice between DNS, RANS, and LES depends on the flow's specific characteristics, the desired detail level, and the available computational resources. Figure~\ref{fig:dns_les_rans} shows a comparison between the different modeling approaches. DNS provides the most accurate results but is computationally expensive, while RANS and LES offer compromises between accuracy and computational cost, making them more practical for many engineering applications.

%----------------------
\section{Time series}
\label{sec:TimeSeries} 
%----------------------

Time Series is a type of data that presents a temporal ordering. It is used in many real-world applications, such as signal processing, finance, weather forecasting, control engineering, communication, human activity recognition, cyber-security, or earthquake prediction. Time series can be described as univariate or multidimensional. Univariate or 1-dimensional time series is an ordered set of real values $X$ of length equal to the number of real values $T$, where $X = [x_1, x_2, ..., x_T]$. While a multidimensional or M-dimensional time series consists of M different univariate time series, where $X = [X^1, X^2, ..., X^M]$ with $X^i \in \mathbb{R}^T$. A Time Series Dataset is defined as $D = \{(X_1, Y_1), (X_2, Y_2), ..., (X_N, Y_N)\}$ as a collection of pairs $(X_i, Y_i)$ where $X_i$ could be a 1-dimensional or M-dimensional time series and an output $Y_i$.

%----------------------
\section{LSTM}
\label{sec:LSTM} 
%----------------------

Recurrent Neural Networks (RNN) are a type of neural network that can keep information about what happened before, they do this with loop connections to their neurons. These neural networks are widely used in speech recognition, language modeling, translation, etc. The main problem with this simple architecture is with “long-term dependencies” when the relevant information happened too long ago. Long Short Term Memory (LSTM) networks are designed to learn long-term dependencies to overcome this issue. The main idea behind the LSTM structure is a memory cell that can accumulate information that can be written and cleared by structures called gates. Fully Connected LSTM (FC-LSTM) is a multivariate version of LSTM, meaning that the input, output, and state are all 1-dimensional vectors.

LSTM neural networks are widely used in Natural Language Processing (NLP), where a long sequence of words in a text needs to be analyzed and used for prediction and classification.

%----------------------
\section{Convolutional Neural Networks}
\label{sec:CNN}
%----------------------

Convolutional Neural Networks (CNN) are a type of neural network primarily designed to process and analyze data in a grid-like organization, especially visual data such as images. Inspired by the organization of the animal visual cortex in the brain, CNNs have been successful in computer vision applications. The key operation in CNNs is the convolution mathematical operation, a specialized kind of linear operation. These networks have a structure called filters, which are applied to the input data to extract features, allowing the neural network to learn hierarchical representations.

CNNs have become the cornerstone of various computer vision tasks, including image classification, object detection, facial recognition, and medical image analysis. Their ability to automatically learn relevant features from raw data makes them particularly effective in tasks where traditional algorithms struggle, such as image understanding and pattern recognition. Additionally, CNNs offer advantages such as parameter sharing, which reduces the number of parameters and enhances model efficiency and translational invariance, enabling robust performance even with variations in object position within an image. Overall, CNNs have revolutionized the field of computer vision and continue to drive advancements in artificial intelligence and image processing applications.

%----------------------
\section{ConvLSTM}
\label{sec:ConvLSTM} 
%----------------------

LSTM neural networks are good for long-time series prediction because they are designed to maintain a context memory of important information that happened long ago as well as recent information. In applications with many dimensions, like spatial data, using an LSTM is inefficient as it contains too much redundancy in the connections between the input. 

ConvLSTM extends the Long Short-Term Memory (LSTM) architecture, incorporating convolutional operations within the LSTM units. It is specifically designed to handle spatiotemporal data, such as video sequences or spatial-temporal patterns in data. ConvLSTM preserves the sequential memory capabilities of LSTM while exploiting the spatial information in data through convolutions. This allows it to capture both temporal dependencies and spatial correlations simultaneously, making it ideal for tasks like video prediction, precipitation nowcasting, and motion tracking. Compared to traditional LSTMs, ConvLSTMs excel in modeling spatial dependencies within sequences, enabling more accurate predictions and better handling of spatially structured data.

%----------------------
\section{Autoencoders}
\label{sec:Autoencoders} 
%----------------------

Autoencoders are an artificial neural network used for unsupervised learning tasks, particularly in dimensionality reduction and data compression. Comprising an encoder and a decoder, they aim to reconstruct input data while learning efficient representations. The encoder compresses the input into a lower-dimensional latent space, while the decoder reconstructs the original data from this representation. Advantages include their ability to learn meaningful representations from unlabeled data, aiding in feature extraction and data denoising. They are widely used in anomaly detection, image denoising, and data generation tasks. Additionally, autoencoders can serve as the foundation for generative models, such as variational autoencoders (VAEs) and generative adversarial networks (GANs), enabling the synthesis of new data samples. Their versatility, simplicity, and capability to learn compact representations make them valuable tools in various domains, including computer vision and natural language processing.



% ============================================================
%
%                   CHAPTER 3: RELATED WORK
%
% =============================================================
\chapter{Related Work}

This chapter presents an overview of existing technologies and research efforts relevant to the development of efficient, on-device language model-based applications. It focuses on three major areas: Retrieval-Augmented Generation (RAG), efficient LLM runtimes like \texttt{llama.cpp}, and lightweight distribution and serving solutions such as Ollama and Llamafile. It also looks at the work done by tinygrad project in attempting to reverse engineer the NPU API.


%----------------------
\section{Retrieval-Augmented Generation (RAG)}
\label{sec:RAG}
%----------------------

Retrieval-Augmented Generation (RAG) is an architectural paradigm that enhances the factual accuracy and contextual relevance of large language models (LLMs) by incorporating external knowledge retrieval into the generation process. Unlike traditional LLMs that rely solely on internal model weights, RAG allows the model to fetch and condition its output on relevant documents retrieved from a corpus \cite{lewis2021retrieval}.

A typical RAG pipeline consists of the following phases:

\begin{enumerate}
    \item \textbf{Query Formulation:} The input query from the user is either used directly or rephrased using prompt engineering techniques to improve retrieval relevance.
    \item \textbf{Retrieval:} A vector store or dense retriever (e.g., using FAISS, Milvus, or ElasticSearch) is queried to obtain top-$k$ semantically relevant documents based on vector similarity.
    \item \textbf{Context Injection:} Retrieved documents are concatenated with the original query and formatted into a prompt.
    \item \textbf{Generation:} The formatted prompt is passed to the language model to generate a context-aware and informative response.
\end{enumerate}

RAG has become a cornerstone for building knowledge-intensive NLP systems, especially in enterprise search, question answering, and summarization tasks \cite{rag_blog, langchain_rag}.


%----------------------
\section{llama.cpp}
\label{sec:llama_cpp}
%----------------------

\texttt{llama.cpp} is a C++ implementation of LLaMA models developed by Meta, optimized for local inference on commodity hardware without GPU requirements. Built upon the GGML tensor library, it provides quantized inference for large models using CPU-friendly formats like 4-bit and 5-bit quantization, making it suitable for running models such as LLaMA, Mistral, and other open-weight transformers on devices ranging from laptops to Raspberry Pi \cite{llamacpp}.

Key features of \texttt{llama.cpp} include:
\begin{itemize}
    \item Highly efficient CPU inference with quantized models.
    \item Cross-platform support (macOS, Linux, Windows).
    \item Integration with popular tooling such as LangChain and Open Interpreter.
    \item Support for multi-threaded inference and memory-mapped model weights for efficient memory usage.
\end{itemize}

This project is a foundation for many desktop LLM applications that require local execution and privacy-aware computation.


%----------------------
\section{Ollama}
\label{sec:ollama}
%----------------------

Ollama is a developer-friendly platform for running LLMs locally with simplified model management and serving. It wraps models like LLaMA 2, Mistral, and Code LLaMA into a streamlined runtime with a CLI and RESTful API, abstracting away hardware-specific setup and providing a plug-and-play experience for developers \cite{ollama}.

Ollama supports:
\begin{itemize}
    \item Running quantized models locally with GPU acceleration where available.
    \item Seamless model downloading and serving.
    \item Custom model creation using a simple Modelfile syntax.
\end{itemize}

It is widely used for prototyping private, offline chatbots and assistants. However, the users need to be technically savvy in cases of issues downloading or running the models. Furthermore, models often may not be quantized and could lead to downloading of model weights in order of many GBs.


%----------------------
\section{Llamafile}
\label{sec:llamafile}
%----------------------


Llamafile, developed by Mozilla-Ocho, enables packaging a complete LLM runtime into a single, self-contained executable file \cite{llamafile}. It leverages the \texttt{llama.cpp} backend and Cosmopolitan Libc to build universal binaries that run across major operating systems (Windows, macOS, Linux) without requiring dependencies.

Notable features:
\begin{itemize}
    \item Distributable as a single file under 1GB (depending on model).
    \item Useful for shipping LLM-based tools with zero-install requirements.
    \item Integrates with web frontends for local chatbot deployment.
\end{itemize}

Llamafile is widely used for prototyping private, offline chatbots and assistants. However, it often requires users to be technically proficient, especially when encountering issues related to downloading or executing models. Moreover, many models are not pre-quantized, potentially resulting in downloads of several gigabytes of model weights.
%----------------------
\section{Tinygrad project and Apple Neural Engine (ANE)}
\label{sec:ANEAPI}
%----------------------

The Apple Neural Engine (ANE) is a custom neural processing unit designed by Apple to accelerate machine learning workloads on its silicon platforms. Introduced with the A11 Bionic chip, the ANE has evolved into a high-performance, low-power DMA-based inference engine embedded in Apple's M-series chips. This section synthesizes insights obtained by the reverse-engineering efforts of the \texttt{tinygrad}~\cite{tinygrad2023ane} project to examine ANE's architecture, capabilities, and compilation flow.

\subsection{Hardware Overview}

The ANE operates primarily as a DMA engine optimized for convolutional operations and supports a wide range of neural network layers and fused operations. Its key hardware features include:

\begin{itemize}
  \item \textbf{16-core architecture:} A 16-wide Kernel DMA engine for parallel computation.
  \item \textbf{5D Tensor Support:} Tensors are structured with width (column), height (row), planes (channels), depth, and group (batch).
  \item \textbf{Supported Data Types:} \texttt{UInt8}, \texttt{Int8}, and \texttt{Float16} (with \texttt{Float32} inputs automatically downcast).
  \item \textbf{Manually Managed 4MB L2 Cache:} Applied only to input/output data; weights are embedded in the compiled program.
  \item \textbf{Execution Unit:} Executes up to 0x300 micro-operations per instruction.
  \item \textbf{Performance:} Approximate 11 TOPS throughput, assuming 32\texttimes32 MAC at ~335 MHz.
\end{itemize}

All memory strides are constrained to multiples of 0x40 bytes, reflecting hardware alignment requirements.

\subsection{Software and Compilation Stack}

The ANE software stack is heavily abstracted behind Apple's proprietary frameworks but has been reverse-engineered to reveal a structured flow (as shown in Fig~\ref{fig:ane-workflow}):
\begin{figure}[h]
    \centering
    \includegraphics[width=0.25\linewidth]{images/ane-workflow.jpg}
    \caption{Apple Neural Engine Workflow}
    \label{fig:ane-workflow}
\end{figure}

\begin{enumerate}
  \item \textbf{Model Definition:} Models are authored in Keras or ONNX.
  \item \textbf{Conversion:} Models are converted to CoreML format using open-source tools such as \texttt{coremltools}.
  \item \textbf{Intermediate Representation:} CoreML is internally converted into \texttt{net.plist} by Apple's Espresso framework.
  \item \textbf{Compilation:} The \texttt{ANECompiler} service transforms \texttt{net.plist} into a hardware-specific binary (.\texttt{hwx}), a Mach-O formatted executable.
  \item \textbf{Execution:} The \texttt{AppleNeuralEngine} and \texttt{ANEServices} handle execution via the kernel extension \texttt{AppleH11ANEInterface}.
\end{enumerate}


\subsection{Instruction Format and Operation Structure}

Each ANE instruction is 0x300 bytes and comprises multiple segments:

\begin{itemize}
  \item \textbf{Header:} Includes DMA addresses and next-op offset.
  \item \textbf{KernelDMASrc:} Specifies weights, bias, and channel usage.
  \item \textbf{Common:} Describes input/output shapes, types, kernel size, and padding.
  \item \textbf{TileDMASrc/TDMADst:} Layout and stride configurations for input/output tensors.
  \item \textbf{L2 and NE:} L2 cache flags and activation parameters.
\end{itemize}

\subsection{Supported Operations and Activations}

ANE supports a variety of operations:

\begin{itemize}
  \item \textbf{Core Ops:} CONV, POOL, EW, CONCAT, RESHAPE, MATRIX\_MULT, TRANSPOSE
  \item \textbf{Advanced:} SCALE\_BIAS, SOFTMAX, INSTANCE\_NORM, BROADCAST, L2\_NORM
  \item \textbf{Fused Ops:} NEFUSED\_CONV, PEFUSED\_POOL, etc.
  \item \textbf{Activations:} RELU, SIGMOID, TANH, CLAMPED\_RELU, PRELU, LOG2/EXP2, CUSTOM\_LUT
\end{itemize}

Over 30 activation functions are supported in hardware.

\subsection{tinygrad Implementation}

The \texttt{tinygrad} project interfaces directly with ANE through a three-stage pipeline:

\begin{itemize}
  \item \textbf{1\_build:} Generates CoreML models using \texttt{coremltools}.
  \item \textbf{2\_compile:} Uses Objective-C and Apple's private ANECompiler framework to compile models into HWX binaries.
  \item \textbf{3\_run:} Loads HWX binaries and executes them on ANE using custom Objective-C wrappers around \texttt{AppleH11ANEInterface}.
\end{itemize}

The implementation also includes tools like \texttt{hwx\_parse.py} for disassembling HWX files and visualizing internal ops.

\subsection{Security and Access}

Execution on ANE requires system entitlements that are typically unavailable to third-party applications:

\begin{itemize}
  \item \texttt{com.apple.ane.iokit-user-access}
  \item Workarounds: amfid patching, kernel extension modification, or use of provisioning profiles.
\end{itemize}

\subsection{Conclusion}

The ANE represents a proprietary, highly optimized inference accelerator that is difficult to access and understand due to Apple’s closed ecosystem. Reverse engineering, as demonstrated by \texttt{tinygrad}, reveals a modular, DMA-centric architecture capable of executing complex neural network operations at high throughput and low latency. As Apple continues to iterate on the ANE, deeper access and tooling may unlock broader ML deployment options on Apple hardware.


% ============================================================
%
%                   CHAPTER 4: METHODS
%
% =============================================================
\chapter{Methodology}
\label{ch:Methodology}
This chapter outlines the technical methodology adopted in the development of the system, covering both low-level design and software implementation. The approach prioritizes performance, privacy, and modularity, leveraging hardware accelerators where possible and maintaining efficient control over data flow and execution.


%----------------------
\section{Application Design - Modules}
\label{sec:ApplicationDesign-Modules}
%----------------------

This section provides an overview of the internal architecture and design principles behind the TLDR desktop application. The application is built with the goal of providing a fast, private, and efficient interface for question-answering and summarization tasks over a user-defined corpus of documents. To achieve this, the design incorporates several performance-aware and hardware-conscious modules, especially tailored for Apple’s M1/M2 architecture.

The TLDR system is structured into modular components, each responsible for a specific functionality in the information processing pipeline. These include embedding generation, context retrieval, vector storage, prompt construction, and output generation via a language model. The workflow between these modules is coordinated to support seamless execution, low latency, and high responsiveness, all while maintaining data privacy by running entirely on-device.


%----------------------
\subsection{Modules Overview}
\label{subsec:modules_overview}
%----------------------
The TLDR application follows a modular architecture where different components are responsible for distinct tasks in the RAG (Retrieval-Augmented Generation) pipeline. Figure~\ref{fig:tldr_modules} illustrates the overall design and the control flow between the modules.

\begin{figure}[H]
    \centering
    \includegraphics[width=1.0\linewidth]{images/tldr-app-modules.jpg}
    \caption{Modules of TLDR application}
    \label{fig:tldr_modules}
\end{figure}

\begin{itemize}
    \item \textbf{User Interface (Swift):} This module provides the graphical frontend for the user. Developed using Swift, it allows users to input queries, view summaries, and interact with the document corpus. All user commands are forwarded to the backend for processing.

    \item \textbf{RAG Backend (lib-tldr):} This is the core orchestrator of the application. It manages the full pipeline, including handling user queries, initiating vector search, performing retrieval, and forwarding context to the language model. It communicates with all supporting modules such as the database, vector search engine, and file system.

    \item \textbf{PostgreSQL Database:} Stores metadata and document indexing information. It ensures efficient retrieval and persistence of preprocessed documents and vector references.

    \item \textbf{File System (Corpus Directory and Vector Dump Files):} Contains the raw document corpus and their corresponding vector dump files. These vectors are accessed using memory-mapped I/O for performance and low memory overhead.


    \item \textbf{NPU Accelerated Cosine Similarity (lib-npu\_accelerator):} Implements hardware-accelerated cosine similarity using Apple's Neural Processing Unit (ANE). The backend invokes this module for fast and parallelizable similarity computation.

    \item \textbf{llama.cpp (lib-llama):} This module is responsible for language generation. Once the RAG backend compiles the context, it is passed to the llama.cpp engine for final response generation.
\end{itemize}


%----------------------
\subsection{File System: Corpus Directory and Vectordump Files}
\label{subsec:FSandVectordump_files}
%----------------------
Vector dump files are binary data structures designed for efficient storage of document embeddings. For each input document, a corresponding vector dump file is created. Each such file contains embedding vectors generated from text chunks along with their corresponding MD5 hashes. This format enables fast similarity search and content verification in document retrieval systems.

%----------------------
\subsubsection{File Structure}
\label{subsec:Vectordump_FileStructure}
%----------------------
The vector dump file follows a sequential binary layout consisting of three main components: a metadata header, followed by embedding data, and finally hash data. This structure allows for efficient random access to embeddings while maintaining data integrity through hash verification. Hash is also further used for fetching the corresponding text chunk after similar vectors are obtained for a query.

   \begin{figure}[H]
    \centering
    \includegraphics[width=1.0\linewidth]{images/VectorDumpFiles.png}
    \caption{Vector dump file structure}
    \label{fig:vectordumpfilestructure}
\end{figure}

%----------------------
\subsubsection{Header}
\label{subsec:Vectordump_Header}
%----------------------
In order to process the vector dump file, the header is first read and the necessary information is obtained regarding the data layout in the file. The information is arranged in the layout as illustrated in Figure~\ref{fig:vectordumpfilestructure}. The elements are as follows:
\begin{itemize}
    \item \texttt{num\_entries} - Total number of embedding/hash pairs stored in the file
    \item \texttt{hash\_size\_bytes} - Size of each MD5 hash in bytes (always 16 for standard MD5)
    \item \texttt{vector\_size\_bytes} - Total byte size of each embedding vector
    \item \texttt{vector\_dimensions} - Number of floating-point dimensions per embedding vector
\end{itemize}

The header is read by pointing the $VectorDumpHeader$ struct to the memory location of the loaded file. This helps obtain the  necessary information required to access the data sections of the file.

%----------------------
\subsubsection{Data Sections}
\label{subsec:Vectordump_DataSections}
%----------------------
Once the header section is loaded, the data obtained is then used for calculating the memory locations of the embedding and hash arrays.
Pointers are used to access these locations to simulate the structure of an array on top of raw binary data read into the memory. This approach is simple and efficient and prevents needless memory allocations and data copies.

\textbf{Embeddings:} Contains $N$ consecutive embedding vectors, where $N$ = \texttt{num\_entries}. Each vector occupies \texttt{vector\_size\_bytes} and represents a \texttt{vector\_dimensions}-dimensional embedding, typically stored as 32-bit floating-point values.

\textbf{Hashes:} Contains $N$ consecutive MD5 hash values, each exactly 16 bytes. However, the smallest unit of storage is of $uint64_t$ type, i.e., units of 2 bytes. Hence, a 16 byte MD5 hash would have a hash size of 8. The hash at index $i$ corresponds to the MD5 digest of the original text chunk used to generate \texttt{embedding[i]}.


%----------------------
\subsubsection{Data Relationship}
\label{subsec:Vectordump_DataRelationship}
%----------------------

The file maintains strict positional correspondence: for any index $i \in [0, N-1]$, \texttt{embedding[i]} and \texttt{hash[i]} represent the same document chunk. This one-to-one mapping enables efficient lookup operations and integrity verification during retrieval.

The data is used as follows:
\begin{enumerate}[label=\arabic*.]
\item Load the first half of the file in memory using \textbf{mmap} and perform cosine similarity search.
    \item Obtain index of the top $K$ relevant vectors from Cosine similarity module.
    \item Fetch the hash values at the obtained indices.
    \item Query the database for text chunks associated with the hash values.
\end{enumerate}
    
This design allows for prioritized access to necessary data and its direct usage for cosine similarity search with no further processing or data manipulations, allowing for an efficient search through the entire corpus.




User Interface RAG Backend  PostgreSQL Database NPU Accelerated Cosine Similarity  {llama.cpp
%----------------------
\subsection{File System: Corpus Directory and Vectordump Files}
\label{subsec:FSandVectordump_files}
%----------------------


%----------------------
\subsection{File System: Corpus Directory and Vectordump Files}
\label{subsec:FSandVectordump_files}
%----------------------



%----------------------
\section{Application Design - Workflow Overview}
\label{subsec:TldrWorkflowOverview}
%----------------------

Figure~\ref{fig:tldr_workflow} illustrates the end-to-end workflow of the TLDR application. It is divided into three primary phases: system initialization, document corpus embedding, and query-based retrieval-augmented generation (RAG).

\subsubsection{System Initialization}

\begin{figure}[H]
    \centering
    \includegraphics[width=1.0\linewidth]{images/tldr-app-worklfow-pt1.jpg}
    \caption{RAG Output evaluation metrics ~\cite{cardenas2023rag}}
    \label{fig:autoregressive_decoding}
\end{figure}


The initialization phase sets up the backend infrastructure and prepares the application for use. This includes:

\begin{itemize}
    \item The user launches the application, triggering the backend.
    \item The \textbf{RAG Backend} initializes a connection pool to the \textbf{PostgreSQL Database}.
    \item The LLM weights and context pools are loaded via \texttt{llama.cpp}, enabling multi-threaded inference.
    \item Cosine similarity routines are prepared via the \textbf{NPU accelerator} module.
    \item System status is communicated back to the \textbf{User Interface}, indicating readiness.
\end{itemize}

\subsubsection{Corpus Embedding}

\begin{figure}[H]
    \centering
    \includegraphics[width=1.0\linewidth]{images/tldr-app-worklfow-pt1.jpg}
    \caption{RAG Output evaluation metrics ~\cite{cardenas2023rag}}
    \label{fig:autoregressive_decoding}
\end{figure}

When a user specifies a directory to embed:

\begin{itemize}
    \item The \textbf{RAG Backend} scans the specified directory for documents using the File System.
    \item Each document is loaded, chunked, and converted to embeddings using a pre-defined embedder.
    \item Embeddings, text chunks, and associated metadata are inserted into the \textbf{PostgreSQL Database}.
    \item In parallel, the backend also writes a vector dump file to the \textbf{File System}, which stores the hash of each vector for quick access.
\end{itemize}

This dual-storage mechanism (DB + mmap vector cache) allows fast retrieval during inference while maintaining queryable metadata.

\subsubsection{Query Execution via RAG}

\begin{figure}[H]
    \centering
    \includegraphics[width=1.0\linewidth]{images/tldr-app-worklfow-pt3.jpg}
    \caption{RAG Output evaluation metrics ~\cite{cardenas2023rag}}
    \label{fig:autoregressive_decoding}
\end{figure}

Once the corpus is embedded, the user may input a query. The following steps are executed:

\begin{itemize}
    \item The user query is embedded using the same embedding model.
    \item The embedded query vector is sent to the \textbf{Cosine Similarity} module running on the NPU.
    \item A top-$k$ similarity search is performed against memory-mapped vector files using the NPU, returning hash values of the best matches.
    \item These hashes are used to retrieve the corresponding text chunks from the \textbf{PostgreSQL Database}.
    \item A prompt is constructed and sent to the \textbf{LLM} via \texttt{llama.cpp}.
    \item The generated response is sent back to the \textbf{User Interface}.
\end{itemize}

This phase exemplifies the retrieval-augmented generation paradigm, grounded entirely in the user’s local corpus and executed with optimized hardware utilization.





%----------------------
\section{III. Application Implementation}
\label{sec:III.ApplicationImplementation}
%----------------------


% ============================================================
%
%                   CHAPTER 5: RESULTS
%
% =============================================================

\chapter{Results}
\label{ch:Results}



\section{Bertscore comparison}
\label{BertscoreComparison}

BERTScore is an evaluation metric for natural language generation tasks that leverages contextual embeddings from pre-trained language models such as BERT. Unlike traditional n-gram based metrics like BLEU or ROUGE, BERTScore computes similarity by aligning tokens in candidate and reference texts using the cosine similarity of their embeddings. This allows it to better capture semantic similarity even when surface forms differ. BERTScore has been shown to correlate strongly with human judgment across various NLP tasks~\cite{zhang2020bertscore}.


We compare sample results from TLDR application using \texttt{Llama-3.2-1B-Instruct-Q3\_K\_L} with \texttt{Chatgpt 4o}

\subsection{Result comparisons}
\label{subsec:Result comparisons}
\begin{itemize}
    \item \textit{97 Things Every Software Architect Should Know}
    \item \textit{Cracking the Coding Interview}
    \item \textit{189 Programming Questions and Solutions}
    \item \textit{System Design Interview: An Insider’s Guide} by Alex Xu
\end{itemize}

\textbf{TLDR Application:}
\\ \textbf{Note:} The references to the sources from TLDR's responses have been trimmed for comparison.

\begin{enumerate}[label=\arabic*.]

    
\item \textbf{User:} What is system design in software engineering?
\\ \textbf{Response:} System design is the process of creating a software system that meets the requirements of the users and the business. It involves defining the system architecture, selecting the appropriate technologies, and designing the user interface. The goal of system design is to create a system that is efficient, scalable, and maintainable. It is a critical process in software engineering that requires careful planning, analysis, and testing.



\item \textbf{User:} how to generate prime numbers?
\\ \textbf{Response:}
To generate prime numbers, you start with a list of all the numbers up to a certain number (in this case, max). Then, you cross off all the multiples of each prime number you find. This process continues until you've crossed off all the multiples of all the prime numbers you've found. The remaining numbers in the list are the prime numbers.
Here's a step-by-step guide:
1. Create a list of all the numbers up to max.
2. Cross off all the multiples of each prime number you find.
3. Continue this process until you've crossed off all the multiples of all the prime numbers you've found.
Note: You can use the Sieve of Eratosthenes algorithm to generate a list of prime numbers up to max.


\item \textbf{User:} what is caching?
\\ \textbf{Response:}
Caching is a technique used to improve the performance of an application by storing frequently accessed data in a temporary storage area, called the cache. This allows the application to retrieve the data quickly, reducing the load on the database or data store. Caching can be used to improve the performance of web applications, database applications, and even mobile applications. It is a technique that can be used to mitigate the problem of the "cold start" problem, where the application has to wait for the data to be retrieved from the database or data store. Caching can be implemented in various ways, including the use of caching tiers, caching strategies, and caching APIs.

\end{enumerate}

\textbf{ChatGPT:}
\\Note: The same documents mentioned earlier are uploaded to ChatGPT with this prompt: 'understand these documents and answer questions that i will ask based on them and keep the answer concise and precise'.

\begin{enumerate}[label=\arabic*.]

\item \textbf{User:} What is system design in software engineering? (use the previously attached documents as reference)
\\ \textbf{Response:} System design in software engineering is the process of defining the architecture, components, data flow, and interfaces of a software system to meet specified requirements. It focuses on how to build scalable, reliable, and maintainable systems by making high-level decisions on technologies, structure, and communication between parts of the system.



\item \textbf{User:} how to generate prime numbers? (use the previously attached documents as reference)
\\ \textbf{Response:}
To generate prime numbers, Cracking the Coding Interview outlines two primary methods: trial division and the Sieve of Eratosthenes. Trial division checks whether a number is divisible by any integer up to its square root and is suitable for verifying if a single number is prime. For generating multiple primes efficiently, the Sieve of Eratosthenes is recommended; it initializes a boolean array representing numbers up to a limit and iteratively marks multiples of each prime as non-prime, resulting in an array of prime indicators. This sieve approach offers better performance with a time complexity of O(n log log n), making it ideal for generating large sets of prime numbers.


\item \textbf{User:} what is caching?  (use the previously attached documents as reference)
\\ \textbf{Response:}
Caching, as described in System Design Interview: An Insider’s Guide and Cracking the Coding Interview, is a technique used to temporarily store frequently accessed or computationally expensive data in fast-access memory (typically RAM) to improve system performance and reduce load on backend systems like databases. It acts as an intermediary layer between clients and data stores, allowing applications to first check the cache for data before querying the database. Common strategies include read-through caching, expiration policies, and eviction methods like Least Recently Used (LRU), with considerations for consistency, fault tolerance, and scalability.

\end{enumerate}
\subsubsection{Quantitative comparison}
\label{subsubsec:tldrresultQuantitativecomparison}
The table below presents the precision, recall, and F1 score obtained using BERTScore \cite{bert-score}. These results are intended as illustrative examples, as the behavior and output of large language models (LLMs) can vary significantly and are not fully captured by a limited set of responses.

\begin{table}[h!]
\centering
\begin{tabular}{|c|c|c|c|}
\hline
\textbf{Row} & \textbf{Precision (P)} & \textbf{Recall (R)} & \textbf{F1 Score} \\
\hline
1 & 0.863623 & 0.897878 & 0.880417 \\
2 & 0.851684 & 0.854167 & 0.852923 \\
3 & 0.919073 & 0.916270 & 0.917669 \\
\hline
\end{tabular}
\caption{Performance metrics for three different evaluations}
\label{tab:metrics}
\end{table}
The results demonstrate a high degree of similarity between the model responses. Overall, the key takeaway is that the TLDR model is capable of generating responses that are comparable in quality to those produced by cloud-based models, despite operating with significantly fewer resources.
\section{Screenshots}
\label{SampleScreenshots}
\begin{figure}[h]
    \centering
    \includegraphics[width=1.0\linewidth]{images/result1.png}
    \caption{TLDR Application  Demonstration screenshot 1}
    \label{fig:tldrmodulesinteraction}
\end{figure}

\begin{figure}[h]
    \centering
    \includegraphics[width=1.0\linewidth]{images/result2.png}
    \caption{TLDR Application Demonstration screenshot 2}
    \label{fig:tldrmodulesinteraction}
\end{figure}

Simple Demo video: https://www.youtube.com/watch?v=KaDgJD-KyKA

% ============================================================
%
%                   CHAPTER 6: CONCLUSION
%
% =============================================================

\chapter{Conclusion}
\label{ch:Conclusion}

% % --------------------------------------
\section{Takeaways}
\label{sec:Takeaways}
% % --------------------------------------
The following are the key takeaways of this project:
\begin{enumerate}[label=\alph*.]
\item Apple Neural Engine (ANE) Utilization: This project explored leveraging the underused NPU for LLM inference, going beyond standard CoreML use cases and showed that it can be viable.

\item Retrieval-Augmented Generation (RAG) without Internet : Implements a lightweight RAG pipeline that only uses on device resources.

\item Portability and Ease of use: Application size of less than 1GB that includes everything, available for download at \href{https://tldr.cool}{\textbf{https://tldr.cool}}

\item Quantized LLMs: Uses compact models (50–500MB) for efficient, on-device inference allowing for seamless multitasking and manages to obtain results comparable to mainstream cloud LLMs.
\end{enumerate}
% % --------------------------------------
\section{Limitations}
\label{sec:Limitations}
% % --------------------------------------
Following are the main limitations uncovered during the implementation of this project:
\begin{enumerate}[label=\alph*.]
\item While this project demonstrates the feasibility to accelerate the retrieval part of the RAG pipeline, the LLM Inference still only leverages the GPU and does not leverage the NPU.

\item Indexing(embedding) takes 90+% of overall runtime

\item Many smaller LLMs available currently (3B or less) are only instruction-tuned i.e trained for text completion and not for chat. This can lead to unfavorable responses during chat.

\item Excessive usage limits to quick draining of power, especially on portable devices.

\item The breadth of retrieval space is limited not by resources but by the context size of the Chat LLM, since both input and expected output must fit in the LLM context.

\end{enumerate}
% % --------------------------------------
\section{Future work}
\label{sec:FutureWork}
% % --------------------------------------
\begin{enumerate}[label=\alph*.]
    \item Enhance data safety mechanisms for vectordump files.
    \item Add NPU backend for GGML and llama.cpp.
    \item Quantize and convert fine-tuned chat-optimized LLMs to the GGUF format.
    \item Implement a dedicated parallelized tokenization module.
    \item Extend the application support beyond Apple Silicon.
    \item Add NPU and GPU acceleration support for SQLite and Postgres vector search extensions (they currently only support CPU).
    \item Create an optimized decoding and tokenization workflow dedicated for embedding (e.g., no KV cache).
\end{enumerate}

\printendnotes

%
% ==========   Bibliography
%
% \nocite{*}   % include everything in the uwthesis.bib file
\bibliographystyle{unsrt}
\bibliography{uwthesis}
%
% ==========   Appendices
%

\end{document}