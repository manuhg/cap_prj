
\chapter {Introduction}
\label{ch:Introduction}

%----------------------
\section{Background}
\label{sec:Background}
%----------------------
Fluid mechanics plays a vital role across various major industries due to the diverse range of technologies and devices that require interaction with fluids for their operation. These industries encompass multiple applications, ranging from aircraft, maritime vessels, and architectural structures to essential medical equipment like ventilators and dialysis machines. For example, in the aviation sector, airplanes navigate through the fluid medium of air externally while internally relying on various fluids such as aviation fuel and hydraulic fluids for proper functioning. Similarly, in sectors like structural engineering, structures like buildings and bridges must be designed to withstand various wind conditions, including hurricanes and tornados. There are several healthcare industrial fluid mechanics applications in the medical realm, from designing medical equipment to disease prevention. Examples of these applications involve modeling blood flow through the human body to identify conditions that may cause health issues and controlling air quality to prevent airborne dispersion and transmission of viruses such as COVID-19. All these examples use fluid mechanics to design a solution.

Modern fluid mechanics solutions use Computational Fluid Dynamics (CFD) simulations to assess engineering challenges. For example, CFD simulations are used in developing innovative airfoil designs for airplanes to enhance efficiency, minimize fuel consumption, and mitigate pollution emissions. To tackle this, it is crucial to solve fluid mechanics models involving large complex systems of equations over large volumes of data. However, there are some challenges in using those simulation techniques. Calculating such fluid dynamics interactions over an aerodynamic system can quickly become a computationally heavy task, especially with the growing complexity of new designs. Significant computation times resulting from complex calculations can drastically increase the development time. As a result, it causes project delays across the various stages of development because it is impossible to perform a fast iterative design process. For example, while creating an airfoil shape, an aeronautical engineer will have to perform several CFD simulations to test the design's performance, implement the necessary modifications to improve the design and test it again, and wait several days for the simulation to run, to continue the work. Having reliable and fast tools to analyze fluid interactions to support an iterative design process strategy is essential to the engineering process. Finding new ways to improve the speed of these simulations can significantly impact the engineering process development time, making it the primary motivation of this work.

Performing CFD simulations can pose a challenge due to the need for substantial computational resources, including processing power and memory. Simulations that are large in scale or have higher grid resolutions may take a long time to complete, leading to a delay in obtaining results. This limitation can hinder the usage of CFD simulations for specific applications or restrict the exploration of design spaces. As a result, it is crucial to develop solutions that can enhance the simulations' execution time. High Performance Computing (HPC) and parallelization techniques are currently used to improve the simulations' execution times and analysis. However, implementing and optimizing these acceleration techniques can be even more challenging than conducting the CFD simulation or the design task for which the simulation is required, consuming research time and computing resources. Although modern high-performance computers provide increased computing power, allowing designers to use CFD simulations instead of real-life experiments, this process is costly or impractical to conduct in many cases. This is especially problematic at the initial design stages of airplanes, boats, bridges, buildings, cars, and other machines and structures interacting with fluids like air or water. 

In recent years, Artificial Intelligence (AI) and Machine Learning (ML) have experienced great success thanks to the improvement of computational capability available to researchers, particularly in the area of Deep Learning (DL) with the application of Artificial Neural Networks (ANN) \cite{lecun_deep_2015}.  It has been used in diverse applications in many areas of science to solve complex problems, for instance, techniques including time series prediction, generative models, dimensionality reduction and principal component analysis, and super-resolution of images and videos. ML and DL techniques have the potential to enchase CFD simulations \cite{vinuesa_enhancing_2022}. Recently, there have been emerging research trends of ML applications for CFDs \cite{vinuesa_emerging_2022} in direct numerical simulation, turbulent modeling, and reduced order models.

This project explores a potential improvement of Computational Fluid Dynamics simulations using Artificial Intelligence to increase the simulation speed while maintaining a low error compared to classical CFD methods.


%----------------------
\section{Research Objective and Solution}
\label{sec:ResearchObjectiveandSolution}
%----------------------
This research aims to employ Deep Learning (DL) methods to create Computational Fluid Dynamics (CFD) simulations involving the interaction of fluid flow with an obstacle in a 2-dimensional environment. The objective is to decrease the simulation's execution time compared to conventional simulation techniques. To achieve this goal, a novel neural network architecture incorporating approaches from existing research in DL for CFD and new ideas for this field are proposed. The DL solution proposed is an end-to-end data-driven solution. This means it is a unified process that can use data to learn the complexities of a fluid flow spatial structure's evolution over time. It can also quickly adapt to changing environments represented by different datasets by directly learning from raw data that represents the intrinsic patterns of fluid mechanics. This solution combines an autoencoder and a generator implemented with a ConvLSTM neural network.

Compared to related solutions in DL for CFD that use different datasets (like Homogeneous Isotropic Turbulence data) or focus on the Dimensionality Reduction techniques of the input data, this research's emphasis is on generating a fluid flow simulation interacting with an obstacle and how well it generalize between distinct shapes of obstacles in various positions and sizes. Additionally, this research studies the effectiveness of the Convolutional LSTM neural network by implementing the model using only this type of architecture.

Two primary metrics are used to evaluate the success of this solution: execution time and accuracy. The execution time of the simulation using the DL model is compared to a traditional CFD simulation. Its accuracy, when compared to the conventional method, is measured using the Mean Squared Error (MSE) (See Equation~\ref{eq:mse}), also known as the Mean Squared Deviation (MSD). The goal is to reduce the execution time while maintaining a good enough accuracy to preserve the pattern structure of the fluid in the generated sequence flow.

\begin{equation}
    \begin{aligned}
        MSE(y, \hat{y}) = \frac{1}{n} \sum_{i=1}^{n}(y_i-\hat{y_i})^2
    \end{aligned}
    \label{eq:mse}
\end{equation}

%----------------------
\section{Scope}
\label{sec:Scope}
%----------------------
The proposed solution focuses on the simple case of a fluid flow interacting with a stationary shape in a 2-dimensional environment and replicates the fluid's behavior using a DL model as accurately as possible while improving the execution time compared to a traditional simulation. 

Because the Navier-Stokes equations used in fluid dynamics are so complex and chaotic (See Section~\ref{ch:TheoreticalBackground}), finding an analytical solution for some problems is impossible. This is why numerical techniques are used to approximate the solutions. Research and industry rely on approximated results to perform their experiments and designs. This means that even when the model solution results are not so precise but provide a fast approximation of the data, it still has value since it is a tool to quickly iterate initial designs that can later be validated with more accurate but slow and resource-demanding methods.

%----------------------
\section{Paper overview}
\label{sec:PaperOverview}
%----------------------
This paper is organized as follows: Chapter~\ref{ch:TheoreticalBackground} explains the main concepts for this work related to Computational Fluid Dynamics and Deep Learning. Chapter~\ref{ch:RelatedWork} presents related work and the current state of DL research for CFD and discusses previous related research relevant to this study. Chapter~\ref{ch:Methods} explains all the methods involved in developing this research and the solution, including the data collection, the DL model architecture and training, and the evaluation techniques. Chapter~\ref{ch:Results} shows the results with its analysis and discussion. Finally, Chapter~\ref{ch:Conclusion} presents the conclusions of this research, its limitations, and future work based on the results obtained.

