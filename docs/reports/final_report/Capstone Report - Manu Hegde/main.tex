\documentclass [11pt, proquest] {uwthesis}[03/03/15] %[2021/06/15]
\usepackage{graphicx}
\usepackage[sort&compress,numbers]{natbib}
\usepackage{multirow}
\usepackage{caption}
\usepackage{subcaption}
\usepackage[breaklinks,hidelinks]{hyperref}
\usepackage{amsmath,amssymb,amsfonts,amsthm,epsfig,epstopdf,titling,url}
\usepackage{enumerate}
\usepackage{algorithm,algorithmic}
\usepackage{{booktabs}}

\newenvironment{packed_enum}{
\begin{enumerate}[a)]
  \setlength{\itemsep}{1pt}
  \setlength{\parskip}{0pt}
  \setlength{\parsep}{0pt}
}{\end{enumerate}}

\newenvironment{packed_enum1}{
\begin{enumerate}[1.]
  \setlength{\itemsep}{1pt}
  \setlength{\parskip}{0pt}
  \setlength{\parsep}{0pt}
}{\end{enumerate}}

\newenvironment{packed_enum_par}{
\begin{enumerate}[1)]
  \setlength{\itemsep}{1pt}
  \setlength{\parskip}{0pt}
  \setlength{\parsep}{0pt}
}{\end{enumerate}}

\newenvironment{packed_e}{
\begin{enumerate}
  \setlength{\itemsep}{1pt}
  \setlength{\parskip}{0pt}
  \setlength{\parsep}{0pt}
}{\end{enumerate}}

\newenvironment{packed_itemize}{
\begin{itemize}
  \setlength{\itemsep}{1pt}
  \setlength{\parskip}{0pt}
  \setlength{\parsep}{0pt}
}{\end{itemize}}

% ==========   Local defs and mods

% ----------- definitions format
\theoremstyle{definition}
\newtheorem{defn}{Definition}
\newtheorem{conj}{Conjecture}[section]
\newtheorem{exmp}{Example}[section]



% ================================
% DOCUMENT
% ================================
\begin{document}

\prelimpages
% \raggedbottom


% ---------------------------

\Title{Project TLDR: Standalone desktop application for question answering and summarization using resource-efficient LLMs}

\Author{Manu Hegde}
\Year{2025}
\Program{Computer Science and Software Engineering}

\Chair{Erika Parsons}{}{School of Science, Technology, Engineering \& Mathematics}
\Signature{Michael Stiber}
\Signature{Shane Steinert-Threlkeld}

\copyrightpage

\titlepage  


\setcounter{page}{-1}
\abstract{

This project presents the design and development of a standalone desktop application that enables offline question answering and summarization over a user-provided document corpus using resource-efficient large language models (LLMs). Targeted to run on Apple's M1 / M2 hardware, it leverages on-device processing through Apple's neural engine (ANE) and Apple metal shaders (GPU). The application aims to address growing concerns about data privacy, resource consumption, and user accessibility. Unlike cloud-based tools, which require constant internet connectivity and expose sensitive data to third-party servers, this application provides a secure, localized alternative optimized for academic researchers and students. Core capabilities include a graphical interface and retrieval-augmented generation (RAG) over user specified corpus, while leveraging only a fraction of system resources to allow for seamless multi-tasking. Evaluation is based on both functional quality (e.g., BERTScore comparison to ChatGPT) and nonfunctional metrics (e.g., memory, CPU usage). The outcome is a practical and efficient tool that allows users to interact with large academic corpora while maintaining system responsiveness and data confidentiality.


}

%
% ----- contents & etc.
%
\tableofcontents
\listoffigures
\listoftables  

%
% ----- acknowledgments
%
\acknowledgments{% \vskip2pc
  % {\narrower\noindent
I would like to express my gratitude to Prof. Erika Parsons for all the valuable guidance and help during this work. Furthermore, I sincerely thank Prof. Steinert-Threlkeld and Prof. Stiber for accepting my request to be on the committee for this thesis and for providing precise feedback.
  % \par}
}
%
% end of the preliminary pages
 
 
 
%
% ==========      Text pages
%
\textpages
% ============================================================
%
%                   CHAPTER 1: INTRODUCTION
%
% =============================================================

\chapter {Introduction}
\label{ch:Introduction}

%----------------------
\section{Background and Motivation}
\label{sec:BackgroundAndMotivation}
%----------------------

The field of Natural Language Processing (NLP) has undergone a significant transformation with the advent of Large Language Models (LLMs), which are capable of performing complex language understanding and generation tasks. Groundbreaking works such as the Transformer architecture \cite{vaswani2017attention}, BERT \cite{devlin2018bert}, and GPT-family models \cite{brown2020language, openai2023gpt4} have paved the way for highly capable models that support applications such as summarization \cite{liu2019text}, question answering \cite{izacard2021leveraging}, and document understanding \cite{beltagy2020longformer}. These advances have been further systematized in the concept of foundation models \cite{bommasani2021opportunities}, which emphasize the broad applicability and adaptability of pre-trained LLMs.

Despite their success, most widely used LLM applications operate via cloud-based services, which introduce significant limitations when it comes to privacy, data security, and control over computational resources. This is particularly concerning in academic contexts, where students and researchers often deal with sensitive or proprietary content. Recent studies have raised awareness of the risks associated with exposing private data to generative models, including membership inference \cite{mattern2023membership} and data extraction attacks \cite{nasr2023extraction}. Moreover, surveys indicate increasing usage of LLMs in research and education, highlighting both the demand for such tools and the concerns around data governance \cite{deschenes2024survey, hosseini2023chatgpt}.

Simultaneously, the hardware landscape has evolved to enable local deployment of such models. Apple's M1 and M2 chipsets integrate high-performance CPUs, GPUs, and a dedicated Neural Processing Unit (NPU) through the Apple Neural Engine (ANE). These architectures offer a promising platform for efficient, on-device inference of LLMs, provided the models are adapted appropriately to operate under limited memory and compute budgets.

This convergence of high-capability models, growing privacy concerns, and increasingly powerful consumer hardware forms the backdrop for \textit{Project TLDR}—a standalone desktop application for summarization and question answering over a user-specified corpus. The tool is designed to run entirely offline, preserving user privacy while leveraging optimized LLM inference. The project makes use of modern techniques such as quantization \cite{jacob2017quantization} and low-rank adaptation (LoRA) \cite{hu2021lora} to reduce computational overhead and improve deployment feasibility on M1/M2 hardware. Additionally, the use of Retrieval-Augmented Generation (RAG) \cite{lewis2020rag} ensures that answers and summaries are grounded in user-provided text, enhancing both contextual relevance and factual consistency.

In essence, this project is motivated by the goal of empowering academic users with a practical, secure, and efficient means of engaging with large volumes of textual data. By tying together advances in NLP, secure computing practices, and consumer-grade hardware acceleration, Project TLDR aims to demonstrate that high-quality language understanding can be brought directly to the user's device—without compromise.
%----------------------
\section{Our Contributions}
\label{sec:OurContributions}
%----------------------

In this project, we present \textit{Project TLDR}, a privacy-preserving, offline, and resource-efficient desktop application that enables users to perform Question Answering (QA) and Summarization over personal document repositories. Designed primarily for MacOS systems powered by Apple’s M1 and M2 architectures, the application aims to support academic and research workflows where confidentiality, simplicity, and efficiency are paramount.

Our key contributions are as follows:

\begin{itemize}
    
    \item \textbf{Novel Utilization of Apple Neural Engine (ANE)}: A significant technical contribution of this project is our investigation into utilizing Apple’s underused Neural Processing Unit (ANE), capable of up to 11 TOPS in INT8 precision \cite{apple2024ane}. While current LLM deployment frameworks such as LLaMA.cpp \cite{llamacpp2023} or Ollama\cite{ollama2023} do not harness this co-processor, we demonstrate and document methods to tap into the ANE for local inference acceleration. We build on the NPU API reverse-engineering work by tinygrad \cite{tinygrad2023ane} and leverage the learnings to open a new direction for efficient LLM deployment on Apple silicon devices. We hence leverage the NPU outside of traditional CoreML model deployment paradigm and demonstrate how it can be used for various use cases.

    \item \textbf{Retrieval-Augmented Generation (RAG) Architecture}: We implement a lightweight yet effective RAG pipeline \cite{lewis2020rag} for performing QA and summarization tasks over local collections of documents. This enables the application to provide context-grounded, source-aware responses from user-specified corpora while leveraging limited compute resources.

    \item \textbf{Efficient On-Device Inference Using Quantized LLMs}: We leverage quantized transformer models \cite{jacob2017quantization}, reducing memory and compute demands without compromising output quality. Instead of multi-gigabyte model downloads (as required by tools like Ollama \cite{ollama2023} or LLaMA.cpp \cite{llamacpp2023}), we use compact models (50–500MB) that support practical usage scenarios with minimal setup, enhancing portability and usability for non-technical users.

    \item \textbf{User-Friendly and Ready-to-Use Design}: Unlike tools such as Ollama \cite{ollama2023} and LLaMAFile\cite{llamafile2023}, which require technical familiarity and understanding the nuances of various models, our application provides a clean graphical interface with ready-to-use capabilities tailored to common academic needs—eliminating the steep learning curve and reducing operational friction.
    
    \item \textbf{Privacy-Preserving Document Analysis}: By running entirely on-device, our application mitigates the risks associated with uploading sensitive or proprietary documents to third-party services (e.g., ChatGPT, Claude, Gemini), which have raised concerns over data leakage \cite{mattern2023membership,nasr2023extraction}. Users can securely summarize, query, and rephrase information without network access or cloud APIs.

\end{itemize}

Through this suite of contributions, \textit{Project TLDR} demonstrates that meaningful and secure LLM-powered applications can be brought directly to end-users without reliance on cloud services or specialized technical knowledge, thereby filling a critical gap in the current LLM applications ecosystem.


%----------------------
\section{Scope}
\label{sec:Scope}
%----------------------
The proposed solution focuses on the simple case of a fluid flow interacting with a stationary shape in a 2-dimensional environment and replicates the fluid's behavior using a DL model as accurately as possible while improving the execution time compared to a traditional simulation. 

Because the Navier-Stokes equations used in fluid dynamics are so complex and chaotic (See Section~\ref{ch:TheoreticalBackground}), finding an analytical solution for some problems is impossible. This is why numerical techniques are used to approximate the solutions. Research and industry rely on approximated results to perform their experiments and designs. This means that even when the model solution results are not so precise but provide a fast approximation of the data, it still has value since it is a tool to quickly iterate initial designs that can later be validated with more accurate but slow and resource-demanding methods.

%----------------------
\section{Paper overview}
\label{sec:PaperOverview}
%----------------------
This paper is organized as follows: Chapter~\ref{ch:TheoreticalBackground} explains the main concepts for this work related to Computational Fluid Dynamics and Deep Learning. Chapter~\ref{ch:RelatedWork} presents related work and the current state of DL research for CFD and discusses previous related research relevant to this study. Chapter~\ref{ch:Methods} explains all the methods involved in developing this research and the solution, including the data collection, the DL model architecture and training, and the evaluation techniques. Chapter~\ref{ch:Results} shows the results with its analysis and discussion. Finally, Chapter~\ref{ch:Conclusion} presents the conclusions of this research, its limitations, and future work based on the results obtained.



% ============================================================
%
%                   CHAPTER 2: THEORETICAL BACKGROUND
%
% =============================================================

\chapter {Theoretical Background}
\label{ch:TheoreticalBackground}

This chapter includes the theoretical background for this work discussing related concepts in Computational Fluid Dynamics and Deep Learning.

%----------------------
\section{Fluid dynamics and Navier-Stokes equations}
\label{sec:FluidDynamicsAndNavier-StokesEquations} 
%----------------------

Fluid dynamics is the branch of physics that studies the motion of fluids, both liquids and gases and their interactions with solid boundaries and other fluids. It encompasses a wide range of phenomena, from the flow of water in rivers and the atmosphere's motion to blood circulation in organisms and the behavior of fluids in engineering systems.

The Navier-Stokes equations (See Equations~\ref{eq:navier-stokes}) are fundamental partial differential equations governing the motion of viscous fluids. They describe how fluid velocity, pressure, density, and viscosity evolve over time in response to external forces. The equations are derived from Newton's second law of motion, conservation of mass, and conservation of momentum principles.

The equations consist of two main components: 1) the continuity equation, which represents the conservation of mass, and 2) the conservation of momentum equation, derived from Newton's second law, which describes how the velocity of the fluid changes in response to external forces and internal forces (pressure and viscosity). This is defined in Equation~\ref{eq:navier-stokes} with $t$ time, $\rho$ density, $u$ velocity, $p$ pressure, $\mu$ viscosity, and $F$ external forces.

\begin{equation}
    \begin{aligned}
        \nabla \cdot u &= 0 \\
        \rho(\frac{\partial u}{\partial t} + (u\cdot\nabla)u) &= -\nabla p + \mu \nabla^2 u + \rho F
    \end{aligned}
    \label{eq:navier-stokes}
\end{equation}

The Navier-Stokes equations are nonlinear, leading to complex and sometimes chaotic behavior, such as turbulence. Despite their simplicity, solving these equations analytically for many practical problems is often impossible, leading to the widespread use of numerical methods.

The Navier-Stokes equations have vast applications in various fields, including engineering, meteorology, oceanography. They support the design of aircraft, ships, and vehicles, studying weather patterns, and understanding fluid flow in pipes and channels. However, many fundamental aspects of these equations, including turbulence, remain unsolved problems in mathematics and physics.

%----------------------
\section{Turbulent flow and Reynolds number}
\label{sec:TurbulentFlowAndReynoldsNumber} 
%----------------------

Turbulent flow is characterized by chaotic and unpredictable fluid motion, with irregular fluctuations in velocity, pressure, and flow patterns. It occurs when inertial forces dominate over viscous forces, leading to mixing and eddy formation. The Reynolds number (Re) is a dimensionless parameter that characterizes the flow regime of a fluid. When Re is high, it is a turbulent flow, indicated by the visual appearance of swirling patterns (or eddies), while low Re values indicate a laminar flow type, recognized by a "smooth" flow of the fluid. Thus, turbulent flow is directly correlated to Reynolds numbers, with higher Re values corresponding to more turbulent behavior and lower Re values indicating laminar flow. Equation~\ref{eq:reynolds-numbers} defines Re, with $\rho$ fluid density, L length scale, U velocity, and $\mu$ viscosity.

\begin{equation}
    Re = \frac{\rho LU}{\mu}
    \label{eq:reynolds-numbers}
\end{equation}

%----------------------
\section{Direct Numerical Simulations, RANS and LES}
\label{sec:DirectNumericalSimulationsRANSAndLES} 
%----------------------

In computational fluid dynamics (CFD), Direct Numerical Simulation (DNS), Reynolds-Averaged Navier-Stokes (RANS), and Large Eddy Simulation (LES) are three common approaches for simulating fluid flows, each with its own set of advantages and limitations.

DNS directly solves the Navier-Stokes equations without any modeling assumptions, providing detailed information on all scales of motion in the flow. However, DNS requires high computational resources and is typically only feasible for relatively low Reynolds number flows due to its high computational cost scaling with the cube of the Reynolds number.

RANS averages the Navier-Stokes equations over time to obtain mean flow quantities and then models the effects of turbulent fluctuations using empirical turbulence models. RANS is computationally less expensive than DNS and is suitable for a wide range of engineering applications. However, RANS relies on turbulence models that introduce modeling errors and uncertainties, particularly for complex flows.

LES resolves large-scale turbulent structures explicitly while modeling the effects of smaller-scale turbulence. It strikes a balance between the accuracy of DNS and the computational cost of RANS, making it suitable for simulating moderately high Reynolds number flows. LES captures the essential features of turbulence while reducing modeling errors compared to RANS.

\begin{figure}[H]
    \centering
    \includegraphics[width=0.4\linewidth]{images/dns_les_rans.png}
    \caption{Comparison between DNS, LES, and RANS modeling}
    \label{fig:dns_les_rans}
\end{figure}

The choice between DNS, RANS, and LES depends on the flow's specific characteristics, the desired detail level, and the available computational resources. Figure~\ref{fig:dns_les_rans} shows a comparison between the different modeling approaches. DNS provides the most accurate results but is computationally expensive, while RANS and LES offer compromises between accuracy and computational cost, making them more practical for many engineering applications.

%----------------------
\section{Time series}
\label{sec:TimeSeries} 
%----------------------

Time Series is a type of data that presents a temporal ordering. It is used in many real-world applications, such as signal processing, finance, weather forecasting, control engineering, communication, human activity recognition, cyber-security, or earthquake prediction. Time series can be described as univariate or multidimensional. Univariate or 1-dimensional time series is an ordered set of real values $X$ of length equal to the number of real values $T$, where $X = [x_1, x_2, ..., x_T]$. While a multidimensional or M-dimensional time series consists of M different univariate time series, where $X = [X^1, X^2, ..., X^M]$ with $X^i \in \mathbb{R}^T$. A Time Series Dataset is defined as $D = \{(X_1, Y_1), (X_2, Y_2), ..., (X_N, Y_N)\}$ as a collection of pairs $(X_i, Y_i)$ where $X_i$ could be a 1-dimensional or M-dimensional time series and an output $Y_i$.

%----------------------
\section{LSTM}
\label{sec:LSTM} 
%----------------------

Recurrent Neural Networks (RNN) are a type of neural network that can keep information about what happened before, they do this with loop connections to their neurons. These neural networks are widely used in speech recognition, language modeling, translation, etc. The main problem with this simple architecture is with “long-term dependencies” when the relevant information happened too long ago. Long Short Term Memory (LSTM) networks are designed to learn long-term dependencies to overcome this issue. The main idea behind the LSTM structure is a memory cell that can accumulate information that can be written and cleared by structures called gates. Fully Connected LSTM (FC-LSTM) is a multivariate version of LSTM, meaning that the input, output, and state are all 1-dimensional vectors.

LSTM neural networks are widely used in Natural Language Processing (NLP), where a long sequence of words in a text needs to be analyzed and used for prediction and classification.

%----------------------
\section{Convolutional Neural Networks}
\label{sec:CNN}
%----------------------

Convolutional Neural Networks (CNN) are a type of neural network primarily designed to process and analyze data in a grid-like organization, especially visual data such as images. Inspired by the organization of the animal visual cortex in the brain, CNNs have been successful in computer vision applications. The key operation in CNNs is the convolution mathematical operation, a specialized kind of linear operation. These networks have a structure called filters, which are applied to the input data to extract features, allowing the neural network to learn hierarchical representations.

CNNs have become the cornerstone of various computer vision tasks, including image classification, object detection, facial recognition, and medical image analysis. Their ability to automatically learn relevant features from raw data makes them particularly effective in tasks where traditional algorithms struggle, such as image understanding and pattern recognition. Additionally, CNNs offer advantages such as parameter sharing, which reduces the number of parameters and enhances model efficiency and translational invariance, enabling robust performance even with variations in object position within an image. Overall, CNNs have revolutionized the field of computer vision and continue to drive advancements in artificial intelligence and image processing applications.

%----------------------
\section{ConvLSTM}
\label{sec:ConvLSTM} 
%----------------------

LSTM neural networks are good for long-time series prediction because they are designed to maintain a context memory of important information that happened long ago as well as recent information. In applications with many dimensions, like spatial data, using an LSTM is inefficient as it contains too much redundancy in the connections between the input. 

ConvLSTM extends the Long Short-Term Memory (LSTM) architecture, incorporating convolutional operations within the LSTM units. It is specifically designed to handle spatiotemporal data, such as video sequences or spatial-temporal patterns in data. ConvLSTM preserves the sequential memory capabilities of LSTM while exploiting the spatial information in data through convolutions. This allows it to capture both temporal dependencies and spatial correlations simultaneously, making it ideal for tasks like video prediction, precipitation nowcasting, and motion tracking. Compared to traditional LSTMs, ConvLSTMs excel in modeling spatial dependencies within sequences, enabling more accurate predictions and better handling of spatially structured data.

%----------------------
\section{Autoencoders}
\label{sec:Autoencoders} 
%----------------------

Autoencoders are an artificial neural network used for unsupervised learning tasks, particularly in dimensionality reduction and data compression. Comprising an encoder and a decoder, they aim to reconstruct input data while learning efficient representations. The encoder compresses the input into a lower-dimensional latent space, while the decoder reconstructs the original data from this representation. Advantages include their ability to learn meaningful representations from unlabeled data, aiding in feature extraction and data denoising. They are widely used in anomaly detection, image denoising, and data generation tasks. Additionally, autoencoders can serve as the foundation for generative models, such as variational autoencoders (VAEs) and generative adversarial networks (GANs), enabling the synthesis of new data samples. Their versatility, simplicity, and capability to learn compact representations make them valuable tools in various domains, including computer vision and natural language processing.



% ============================================================
%
%                   CHAPTER 3: RELATED WORK
%
% =============================================================
\chapter{Related Work}

This chapter presents an overview of existing technologies and research efforts relevant to the development of efficient, on-device language model-based applications. It focuses on three major areas: Retrieval-Augmented Generation (RAG), efficient LLM runtimes like \texttt{llama.cpp}, and lightweight distribution and serving solutions such as Ollama and Llamafile. It also looks at the work done by tinygrad project in attempting to reverse engineer the NPU API.


%----------------------
\section{Retrieval-Augmented Generation (RAG)}
\label{sec:RAG}
%----------------------

Retrieval-Augmented Generation (RAG) is an architectural paradigm that enhances the factual accuracy and contextual relevance of large language models (LLMs) by incorporating external knowledge retrieval into the generation process. Unlike traditional LLMs that rely solely on internal model weights, RAG allows the model to fetch and condition its output on relevant documents retrieved from a corpus \cite{lewis2021retrieval}.

A typical RAG pipeline consists of the following phases:

\begin{enumerate}
    \item \textbf{Query Formulation:} The input query from the user is either used directly or rephrased using prompt engineering techniques to improve retrieval relevance.
    \item \textbf{Retrieval:} A vector store or dense retriever (e.g., using FAISS, Milvus, or ElasticSearch) is queried to obtain top-$k$ semantically relevant documents based on vector similarity.
    \item \textbf{Context Injection:} Retrieved documents are concatenated with the original query and formatted into a prompt.
    \item \textbf{Generation:} The formatted prompt is passed to the language model to generate a context-aware and informative response.
\end{enumerate}

RAG has become a cornerstone for building knowledge-intensive NLP systems, especially in enterprise search, question answering, and summarization tasks \cite{rag_blog, langchain_rag}.


%----------------------
\section{llama.cpp}
\label{sec:llama_cpp}
%----------------------

\texttt{llama.cpp} is a C++ implementation of LLaMA models developed by Meta, optimized for local inference on commodity hardware without GPU requirements. Built upon the GGML tensor library, it provides quantized inference for large models using CPU-friendly formats like 4-bit and 5-bit quantization, making it suitable for running models such as LLaMA, Mistral, and other open-weight transformers on devices ranging from laptops to Raspberry Pi \cite{llamacpp}.

Key features of \texttt{llama.cpp} include:
\begin{itemize}
    \item Highly efficient CPU inference with quantized models.
    \item Cross-platform support (macOS, Linux, Windows).
    \item Integration with popular tooling such as LangChain and Open Interpreter.
    \item Support for multi-threaded inference and memory-mapped model weights for efficient memory usage.
\end{itemize}

This project is a foundation for many desktop LLM applications that require local execution and privacy-aware computation.


%----------------------
\section{Ollama}
\label{sec:ollama}
%----------------------

Ollama is a developer-friendly platform for running LLMs locally with simplified model management and serving. It wraps models like LLaMA 2, Mistral, and Code LLaMA into a streamlined runtime with a CLI and RESTful API, abstracting away hardware-specific setup and providing a plug-and-play experience for developers \cite{ollama}.

Ollama supports:
\begin{itemize}
    \item Running quantized models locally with GPU acceleration where available.
    \item Seamless model downloading and serving.
    \item Custom model creation using a simple Modelfile syntax.
\end{itemize}

It is widely used for prototyping private, offline chatbots and assistants. However, the users need to be technically savvy in cases of issues downloading or running the models. Furthermore, models often may not be quantized and could lead to downloading of model weights in order of many GBs.


%----------------------
\section{Llamafile}
\label{sec:llamafile}
%----------------------


Llamafile, developed by Mozilla-Ocho, enables packaging a complete LLM runtime into a single, self-contained executable file \cite{llamafile}. It leverages the \texttt{llama.cpp} backend and Cosmopolitan Libc to build universal binaries that run across major operating systems (Windows, macOS, Linux) without requiring dependencies.

Notable features:
\begin{itemize}
    \item Distributable as a single file under 1GB (depending on model).
    \item Useful for shipping LLM-based tools with zero-install requirements.
    \item Integrates with web frontends for local chatbot deployment.
\end{itemize}

Llamafile is widely used for prototyping private, offline chatbots and assistants. However, it often requires users to be technically proficient, especially when encountering issues related to downloading or executing models. Moreover, many models are not pre-quantized, potentially resulting in downloads of several gigabytes of model weights.
%----------------------
\section{Tinygrad project and Apple Neural Engine (ANE)}
\label{sec:ANEAPI}
%----------------------

The Apple Neural Engine (ANE) is a custom neural processing unit designed by Apple to accelerate machine learning workloads on its silicon platforms. Introduced with the A11 Bionic chip, the ANE has evolved into a high-performance, low-power DMA-based inference engine embedded in Apple's M-series chips. This section synthesizes insights obtained by the reverse-engineering efforts of the \texttt{tinygrad}~\cite{tinygrad2023ane} project to examine ANE's architecture, capabilities, and compilation flow.

\subsection{Hardware Overview}

The ANE operates primarily as a DMA engine optimized for convolutional operations and supports a wide range of neural network layers and fused operations. Its key hardware features include:

\begin{itemize}
  \item \textbf{16-core architecture:} A 16-wide Kernel DMA engine for parallel computation.
  \item \textbf{5D Tensor Support:} Tensors are structured with width (column), height (row), planes (channels), depth, and group (batch).
  \item \textbf{Supported Data Types:} \texttt{UInt8}, \texttt{Int8}, and \texttt{Float16} (with \texttt{Float32} inputs automatically downcast).
  \item \textbf{Manually Managed 4MB L2 Cache:} Applied only to input/output data; weights are embedded in the compiled program.
  \item \textbf{Execution Unit:} Executes up to 0x300 micro-operations per instruction.
  \item \textbf{Performance:} Approximate 11 TOPS throughput, assuming 32\texttimes32 MAC at ~335 MHz.
\end{itemize}

All memory strides are constrained to multiples of 0x40 bytes, reflecting hardware alignment requirements.

\subsection{Software and Compilation Stack}

The ANE software stack is heavily abstracted behind Apple's proprietary frameworks but has been reverse-engineered to reveal a structured flow (as shown in Fig~\ref{fig:ane-workflow}):
\begin{figure}[h]
    \centering
    \includegraphics[width=0.25\linewidth]{images/ane-workflow.jpg}
    \caption{Apple Neural Engine Workflow}
    \label{fig:ane-workflow}
\end{figure}

\begin{enumerate}
  \item \textbf{Model Definition:} Models are authored in Keras or ONNX.
  \item \textbf{Conversion:} Models are converted to CoreML format using open-source tools such as \texttt{coremltools}.
  \item \textbf{Intermediate Representation:} CoreML is internally converted into \texttt{net.plist} by Apple's Espresso framework.
  \item \textbf{Compilation:} The \texttt{ANECompiler} service transforms \texttt{net.plist} into a hardware-specific binary (.\texttt{hwx}), a Mach-O formatted executable.
  \item \textbf{Execution:} The \texttt{AppleNeuralEngine} and \texttt{ANEServices} handle execution via the kernel extension \texttt{AppleH11ANEInterface}.
\end{enumerate}


\subsection{Instruction Format and Operation Structure}

Each ANE instruction is 0x300 bytes and comprises multiple segments:

\begin{itemize}
  \item \textbf{Header:} Includes DMA addresses and next-op offset.
  \item \textbf{KernelDMASrc:} Specifies weights, bias, and channel usage.
  \item \textbf{Common:} Describes input/output shapes, types, kernel size, and padding.
  \item \textbf{TileDMASrc/TDMADst:} Layout and stride configurations for input/output tensors.
  \item \textbf{L2 and NE:} L2 cache flags and activation parameters.
\end{itemize}

\subsection{Supported Operations and Activations}

ANE supports a variety of operations:

\begin{itemize}
  \item \textbf{Core Ops:} CONV, POOL, EW, CONCAT, RESHAPE, MATRIX\_MULT, TRANSPOSE
  \item \textbf{Advanced:} SCALE\_BIAS, SOFTMAX, INSTANCE\_NORM, BROADCAST, L2\_NORM
  \item \textbf{Fused Ops:} NEFUSED\_CONV, PEFUSED\_POOL, etc.
  \item \textbf{Activations:} RELU, SIGMOID, TANH, CLAMPED\_RELU, PRELU, LOG2/EXP2, CUSTOM\_LUT
\end{itemize}

Over 30 activation functions are supported in hardware.

\subsection{tinygrad Implementation}

The \texttt{tinygrad} project interfaces directly with ANE through a three-stage pipeline:

\begin{itemize}
  \item \textbf{1\_build:} Generates CoreML models using \texttt{coremltools}.
  \item \textbf{2\_compile:} Uses Objective-C and Apple's private ANECompiler framework to compile models into HWX binaries.
  \item \textbf{3\_run:} Loads HWX binaries and executes them on ANE using custom Objective-C wrappers around \texttt{AppleH11ANEInterface}.
\end{itemize}

The implementation also includes tools like \texttt{hwx\_parse.py} for disassembling HWX files and visualizing internal ops.

\subsection{Security and Access}

Execution on ANE requires system entitlements that are typically unavailable to third-party applications:

\begin{itemize}
  \item \texttt{com.apple.ane.iokit-user-access}
  \item Workarounds: amfid patching, kernel extension modification, or use of provisioning profiles.
\end{itemize}

\subsection{Conclusion}

The ANE represents a proprietary, highly optimized inference accelerator that is difficult to access and understand due to Apple’s closed ecosystem. Reverse engineering, as demonstrated by \texttt{tinygrad}, reveals a modular, DMA-centric architecture capable of executing complex neural network operations at high throughput and low latency. As Apple continues to iterate on the ANE, deeper access and tooling may unlock broader ML deployment options on Apple hardware.


% ============================================================
%
%                   CHAPTER 4: METHODS
%
% =============================================================

\chapter{Methods}
\label{ch:Methods}
This chapter covers the procedures involved in this research, from data collection to the proposed DL model's architecture and its training, how it is evaluated, and all the tools used during this study. It provides an explanation of how the solution for this study works and its justification.

%----------------------
\section{Data Collection}
\label{sec:DataCollection}
%----------------------
The dataset for this study consists of fluid flows interacting with an obstacle. Its purpose is to train and test the neural network solution proposed in this research. The fluid flows are generated using a numerical method, as is typically done in computational fluid dynamics simulations. The sequences represent the evolution of the fluid flow in space and time; therefore, this dataset could be considered two-dimensional Time Series data (See Section~\ref{sec:TimeSeries}).

Each fluid flow is a sequence of $400$ frames representing the state at a given time. The frames are a two-dimensional grid discretization of the simulated space, with a resolution of 200-width by 100-height cells. The values in the cells represent either the fluid's velocity at the position or -1 if it belongs to the obstacle. 

All the fluid flows in the dataset have a Reynolds number of 220 and flow from left to right, interacting with an obstacle with several dimensions and positions. These obstacles are either circumferences or ellipses, as shown in Figure~\ref{fig:cfd_obstacles}. Circumferences are simple shapes to model and are usually used to demonstrate fluid flow simulations. In contrast, ellipses are an easy simplification of an aircraft wing cross-section with different dimensions and inclinations (known as the ``angle of attack").

\begin{figure}[!h]
    \centering
    \includegraphics[width=0.9\linewidth]{images/cfd_obstacle_examples.png}
    \caption{Two CFD frames from different sequences showing the (a)circumference and the (b)ellipse obstacles}
    \label{fig:cfd_obstacles}
\end{figure}

In aeronautics, several pre-defined airfoils shapes (NACA airfoils) have been studied for the design of aircraft wings \cite{abbott_ira_h_summary_1945}; in particular, the NACA 2412 is used in the popular Cessna 172 Skyhawk. Figure~\ref{fig:naca_airfoils} shows examples of those airfoils. However, these shapes are very complex to calculate. As an alternative for this work, an ellipse shape was chosen as an approximation to the NACA airfoils. The ellipse geometry has a smooth, curved shape that can resemble the streamlined profile of the airfoil. It can provide a good approximation for the leading and trailing edges, capturing the essential aerodynamic characteristics while simplifying the complex geometry of the airfoil. This approximation is particularly useful for preliminary design and analysis, where exact precision is less critical. It allows for easier mathematical manipulation and analysis compared to the exact airfoil shape.

\begin{figure}[!ht]
    \centering
    \includegraphics[width=0.6\linewidth]{images/naca_airfoils.png}
    \caption{The historical evolution of airfoil sections from 1908-1944. Credit: NACA-NASA}
    \label{fig:naca_airfoils}
\end{figure}

\subsection{Definitions}
The fluid flow sequence data can be represented in the following mathematical way. Data is taken from velocity observations in a spatial region on a $W \times H$ grid, which consists of $W$ columns and $H$ rows. Each grid cell has a velocity measurement that varies over time $T$.

\begin{defn}
    \label{defn:T}
    Let $T$ be a finite period of time and $t_i \in T$, where $i \in \{0,1,..., T\}$, is a time instance.
\end{defn}

We can then mathematically represent a fluid flow sequence as follows:

\begin{defn}
    \label{defn:fluid_flow_sequence}
    A fluid flow sequence is represented by a tensor $X \in \mathbb{R}^{T \times W \times H}$, where $x\in X$ and $x \in \mathbb{R}^{W \times H}$ is a velocity observation or fluid state at any given time $t \in T$.
\end{defn}

\begin{defn}
    \label{defn:window}
    Let $\mathcal{W}$ be a window of time $\subseteq T$, where $\mathcal{W}$ is of size $m$ and $1 < m < T$.
\end{defn}

When observations of $x \in X$ are recorded periodically over a time period $T$, we can think of them as a sequence of frames $x_0, x_1, x_2, ..., x_i, ..., x_T$. For this research, the spatiotemporal sequence generation problem is to generate the next most likely frame $x_i$ observation, given a window of previous observations $x_{i-m}, ..., x_{i-2}, x_{i-1}$. The problem can be formalized by equation~\ref{eq:generation-problem}, where $g$ represents the \textit{generate} function performed by the model.

\begin{equation}
    x_i = g(x_{i-m}, ..., x_{i-2}, x_{i-1}) 
    \label{eq:generation-problem}
\end{equation}

% -------------------------------------------------------
\section{Data Preparation}
\label{sec:DataPreparation}
% %-------------------------------------------------------
Before the data is used to train and evaluate the model, the following preprocessing steps are applied to transform the data: 1) data normalization, 2) data slicing, and 3) dividing the dataset into a training and testing set.

\begin{enumerate}
    \item \textbf{Data normalization:} First, the data is normalized by scaling the velocity values between 0 and 1. This data normalization improves the gradient descent optimization during the neural network's training, which is a common requirement for deep learning methods. This scaling is done according to Equation~\ref{eq:min-maxscaling} below.

        \begin{equation}
            v_{scaled} = \frac{v-v_{min}}{v_{max} - v_{min}}
            \label{eq:min-maxscaling}
        \end{equation}
    
    This scaling technique provides robustness to very small standard deviations in the dataset's velocity values. During this process, the obstacle cells are left with a value of -1 to distinguish them from the fluid while maintaining the velocity values in the 0 to 1 range.
    
    \item \textbf{Data slicing:} This step focuses on creating simulated sub-sequences to train the model. The input and output datasets are generated using the original dataset, which consists of the simulated [fluid] sequences. The model takes as input a specific window $\mathcal{W}$ consisting of $m$ frames from the simulated sequence of size $T$; it then uses this window to generate the next frame. Since $m<T$, the length of the input dataset elements must be reduced to match the this window. To do this, each original sequence is segmented into sub-sequences of length $m$ (size of $\mathcal{W}$). After segmenting the original sequence of size $T$, it results in more than one sub-sequence of size $m$ in the input dataset.
    
    Figure~\ref{fig:data_slicing} illustrates this process, where the input dataset to the model (the set of fluid sub-sequences $X$ of size $m$ over a time window $\mathcal{W}$ using definition~\ref{defn:window}), will be used to generate the output set (the next set of frames). During the training step, the model will learn how to infer $x_t$ from the previous $m$ frames (see  Equation~\ref{eq:generation-problem}).
 
    \begin{figure}[H]
        \centering
        % \includegraphics[width=0.9\linewidth]{images/data_slicing.png}
        \includegraphics[width=0.9\linewidth]{images/data_slicing_v2.png}
        \caption{Slicing of sequence into $\mathcal{W}$ sub-sequences used to train the model.}
        \label{fig:data_slicing}
    \end{figure}

    \item \textbf{Training, Validation, and Testing sets:} Finally, the dataset is shuffled and then randomly divided into training, validation, and testing sets with an 8:1:1 ratio, respectively. This is done to validate and test the model using the cross-validation method with samples not used during training. This aims to provide an unbiased evaluation of the model's efficacy, which -- ideally --  should appropriately generalize (for new data) without over-fitting (training data). 
    
\end{enumerate}


%----------------------
\section{Model Architecture}
\label{sec:ModelArchitecture}
%----------------------
The DL solution proposed in this research is an end-to-end model, meaning it will perform all the tasks from data input to generating the fluid flow simulation output in one unified process. In contrast, other related research uses machine-learning or DL techniques for only a specific part of the simulation, leaving the rest to a classical numerical method and making the process more complex. A benefit of our simple approach with only one task is eliminating the need for complex pipelines between separate parts in the simulation process, thus reducing development time and potential sources of error. Additionally, end-to-end models can better adapt to diverse datasets and changing environments since they learn directly from raw data, capturing intricate patterns and relationships that might be missed in traditional approaches, like DNS.

The goal of this model is to generate predictions of a fluid flow's evolution. To accomplish this, the model looks at past states in the flow and generates the following future state. The future state can then be used as an input to continue generating new states in the simulation. This step can be repeated as many times as necessary as a feedback loop shown in Figure~\ref{fig:feedback_loop} below. As a result of this feedback loop, the model can produce a long fluid flow sequence. In Figure~\ref{fig:feedback_loop}.a we can see how the resulting frame is put at the end of the input sequence to generate a new frame (see Section~\ref{subsec:Generator}). Figure~\ref{fig:feedback_loop}.b shows how the model looks at a certain window of the frame and moves forward in time to generate the rest of the sequence. At the beginning of a simulation, the model uses an initial condition or ``ground truth" represented by a number of frames equal to the sliding window ($\mathcal{W}$). As $\mathcal{W}$ slides, new frames are generated, which are in turn used to generate more subsequent frames. Eventually, an entire sequence can be generated, knowing only the first frames from the initial fluid's condition, and the rest are completely generated by the model.

\begin{figure}[!htbp]
    \centering
    \includegraphics[width=1\linewidth]{images/feedback_loop.png}
    \caption{Diagram showing a) the flow between the model's main components and b) sliding window approach for prediction.}
    \label{fig:feedback_loop}
\end{figure}

Because of the data's spatiotemporal characteristics, the neural network has to be capable of analyzing and learning the evolution of fluid flows in two dimensions: space and time. A CNN (See Section~\ref{sec:CNN}) can help understand the spatial structure of the fluid to extract the flow's features and patterns in space. Additionally, the model needs to ``remember" what happened in the past to produce the next state, so it needs a memory mechanism that can be provided by a recurrent neural network such as an LSTM (See Section~\ref{sec:LSTM}), commonly used in Natural Language Processing tasks. As mentioned before, these two types of neural networks have been combined to create the ConvLSTM (See Section~\ref{sec:ConvLSTM}) as an extension of the LSTM network that can also ``look" for features in space and time. For this reason, the ConvLSTM network was chosen to implement the neural network architecture proposed in this study.

Because this data has many dimensions and complexities, a dimensionality reduction is applied to capture the principal components of the flow before generating the next frame. This ensures that the model will rely on a minimal representation of the fluid flow that accurately describes its behavior. For this model architecture, the dimensionality reduction is implemented by an Autoencoder (See Section~\ref{sec:Autoencoders}) neural network that can produce it as part of the same model.

\begin{figure}[!htbp]
    \centering
    \includegraphics[width=0.8\linewidth]{images/model_schematic.png}
    \caption{Schematic diagram of the model architecture}
    \label{fig:model_schematic}
\end{figure}

In summary, the model has two main parts, as shown in Figure~\ref{fig:model_schematic}: 1) an Encoder that reduces the data's dimensionality and 2) a Generator that gets a representation of the past frames and creates the next one.

\subsection{Encoder}
\label{subsec:Encoder}
The Autoencoder is a neural network architecture composed of an Encoder and a Decoder, and it is used for dimensionality reduction (See section~\ref{sec:Autoencoders}). It takes data with many dimensions and creates a representation of this data in a lower-dimensional space. It is used similarly to classical statistical methods, such as singular value decomposition or principal component analysis. As explained in the related works chapter~\ref{ch:RelatedWork}, autoencoders were previously used for fluid flow analysis, such as identifying the main components in a fluid flow and identifying eddies in the flows. It has also been used for dimensionality reduction for other methods that are not deep learning models. 


\subsection{Generator}
\label{subsec:Generator}
The generator takes the lower-dimensional representation of the fluid flow state and generates a new frame in the sequence. Using a lower representation of the data instead of all the original dimensions makes this work easier because the generator will get as input the main components that can describe the flow. 


%----------------------
\section{Model training}
\label{sec:ModelTraining}
%----------------------
The Encoder is trained in conjunction with the Decoder component, which takes the lower dimensionality representation and tries to reconstruct the original input. If the decoder can reconstruct the original data using the reduced representation from the encoder, this means that the representation captures the main elements of the sequence, and the encoder works correctly. This decoder is auxiliary and discarded once the model’s training is completed.

Cross-validation \cite{bengio_practical_2012} is used to ensure that the model performance generalizes well to unseen data during training. This method, commonly used in machine learning, involves splitting the dataset into three parts: a training set, a validation set, and a testing set. The training set is used to train the model, and simultaneously, the validation set is used to evaluate the model's performance during training. Once the model is trained and optimized, it is tested on a separate and unseen test dataset to assess its generalization performance. This technique also helps prevent overfitting by providing an independent dataset for model evaluation. It ensures that the model's performance estimates are more reliable and indicate its performance on new data.

The neural network training job was distributed across the two GPUs available on the server. This was done using the Mirrored Strategy, a synchronous data parallelism algorithm for neural network training. This algorithm replicates the model on both GPUs and splits the dataset between devices. The CPU prepares and sends the data batches to the GPUs. During training, each GPU performs a forward pass over the model on different input data to compute the loss function; subsequently, gradients are calculated on each device based on that loss. Both sets of gradients are then combined with an all-reduce operation by averaging them and re-distributing them across devices to update the model parameters on each replica, synchronizing them. This process is repeated for each data batch for every training epoch.


%----------------------
\section{Hyperparameter Optimization}
\label{sec:HyperparameterOptimization}
%----------------------
The model architecture and training have specific characteristics or hyperparameters that can take various possible values. These hyperparameters impact the model's performance, i.e., different combinations of such values may result in different model efficacy; consequently, it is important to find a suitable hyperparameter combination to optimize the model efficacy.

Hyperparameter optimization with Random Grid Search \cite{bengio_practical_2012} in ML involves systematically exploring a predefined hyperparameter space by randomly sampling combinations of hyperparameters, rather than exhaustively searching through all possible such combinations. This approach helps efficiently find an optimal set of hyperparameters for the model by balancing computational resources and exploring the parameter space. Random grid search randomly selects hyperparameter values from specified distributions and evaluates each combination using cross-validation to determine the set that yields the best performance metric. This technique can effectively search a large hyperparameter space, improving model performance without exhaustive computational costs.

The hyperparameters considered for this model are:

\begin{itemize}
    \item Learning rate: it determines the update step size of the weight in the neural network at each iteration during the optimization process, i.e., the loss function moving towards a minimum. A high learning rate can lead to rapid convergence at the risk of overshooting the optimal solution, causing the model to diverge. On the other hand, a low learning rate ensures steady convergence at the sake of a slower training process which can get stuck in local minima. This parameter significantly impacts the efficiency and effectiveness of the model training. The best training results were achieved with a learning rate of 0.001.
    
    \item Number of layers: the number of neural network layers affects the \textit{capacity} of the model to learn complex representations. In the context of this research, more layers can enable the model to capture more hierarchical features and intricate patterns in the fluid-flows. This can improve the model's performance. However, having more layers makes the model more complex and more susceptible to overfitting the training data. The model achieved the best results using 3 ConvLSTM layers on each Encoder and Decoder component and 4 ConvLSTM layers in the Generator component.
    
    \item Number of filters or kernels on each layer: convolutional filters detect spatial features in the input data. These filters enable the model to capture both spatial and temporal dependencies in the fluid flow. The number of filters impacts the model's ability to extract relevant features. Having more filters can capture more complex data patterns but at the cost of increasing computational cost. This parameter affects the accuracy and efficiency of the model. In the ConvLSTM layers of the Autoencoder, this model architecture got the best results using 64 filters in the first and last layers and 32 in the other ones. In the Generator component, the first layer uses 32 filters and 64 in the rest.
    
    \item Size of the filter: this refers to the convolutional filters' dimension to capture features of the input data. The filter size determines the scope of the local spatial region examined by the model. Larger filters can capture broader spatial patterns but may miss fine-grained details, while smaller filters focus on more localized features, potentially overlooking broader context. This impacts the model's ability to detect relevant features. The following filter sizes in the ConvLSTM layers of the architecture were used to get the best results, in order from the first to last layer of each component: the Encoder has filter sizes of 4 by 4, 3 by 3, and 2 by 2; the Decoder has filters of 2 by 2, 3 by 3, and 4 by 4; and the Generator filter sizes are 3 by 3 for all of its layers. 
\end{itemize}

In order to define a search space of hyperparameter combinations, a range of values is set for each of these hyperparameters.  Potential combinations are selected by random sampling, then a model is trained using those combinations, and the one with the best performance is selected. Once the best combination is found, the model is trained during more epochs to get the final version of the model.

When choosing the $m$ size of the window $\mathcal{W}$, several aspects were considered, e.g., the amount of ``historical'' data used to generate a prediction. Similar to the Exponential Moving Average (EMA) heuristic \cite{kalekar_time_2004} \cite{ugli_cognitive_2023}, which can be used in time-series forecasting scenarios and LSTM models to weight the amount and relevance of historic measurements vs. predicted ones. In our case, we can think of $m$ as the amount of historical information that the model will receive to make a future prediction. In this sense, using a small window will not give enough information to the model. On the other hand, having a bigger window size provides more information and is better for the model, but if $m$ is too big, it will need too much historical information, making the model pointless. Additionally, a bigger window expands the input dimensions, so more computations would be required to compress the data to create the lower dimensional representation. This increase in computations makes the model slower to execute and train, which given our limited computational resources, could render our model impractical. In addition, since the sequences have 400 frames each, the window size $m$ has to be a divisor of 400 to split them into sub-sequences for data preparation, as explained in \ref{sec:DataPreparation}. Overall, since a window of size 2 would be too small to provide the model with enough information, a window size of 4 was chosen since it is the smallest $m$ that allows for the even division of sub-sequences while keeping the model practical.


%----------------------
\section{Proposed Architecture Details}
\label{sec:ArchitectureDetails}
%----------------------

The final model architecture has two components: the \textbf{Autoencoder} and the \textbf{Generator}. The Autoencoder is divided into the Encoder and Decoder. The window $\mathcal{W}$ of $m=4$ frames is first input into the Decoder, which creates a low-resolution representation of the simulation. Next, this low-resolution representation is input into the Generator to create the next frame state of the fluid flow. All these components are implemented in the neural network by a set of ConvLSTM layers followed by either MaxPooling or UpSampling layers. ConvLSTM layers extract features in the data by using convolutional operators called filters or kernels. MaxPooling and UpSampling layers compress or uncompress the intermediate data outputs between ConvLSTM layers. MaxPooling downsamples the input by taking the maximum value in the kernel window along the spatial dimension. UpSampling layers resize the input by interpolating its values. Figure~\ref{fig:model_detail} shows a detailed diagram of this architecture where we can see the order of these layers. In the Encoder, successive layers of ConvLSTM and MaxPooling layers downsample the data, resulting in a representation that is half the input size. This representation is then reconstructed in the Decoder by upsampling it using UpSampling and ConvLSTM layers. Each of these layers has a different amount of filters or pool windows of different sizes, respectively. 
Figure~\ref{fig:model_detail} also shows the output dimension of each layer.

\begin{figure}[!htbp]
    \centering
    \makebox[\textwidth][c]{\includegraphics[width=1.3\textwidth]{images/architecture_detail.png}}
    % \includegraphics[width=1\linewidth]{images/architecture_detail.png}
    \caption{Detail diagram of the model architecture}
    \label{fig:model_detail}
\end{figure}

%----------------------
\section{Evaluation Methods}
\label{sec:EvaluationMethods}
%----------------------
The model evaluation is done using three strategies:
\begin{enumerate}
    \item Calculating the Mean Square Error between the original and generated simulations.
    \item Visually inspecting the simulation result by rendering an image with the generated data and comparing it with the expected data (the ground truth).
    \item Creating a histogram of frame values for the original and the generated data, these will be compared ``side-by-side" to analyze similarities and identify possible drastic differences.
\end{enumerate}

%----------------------
\section{Software and Tools}
\label{sec:SoftwareandTools}
%----------------------
For the implementation of all the methods described in the previous section, the Python \cite{python} scripting language with the following libraries: TensorFlow \cite{tensorflow} and Keras \cite{keras}, for the neural network implementations and training; Numpy \cite{numpy} and scikit-learn \cite{scikit-learn}, for data manipulation and preprocessing; and Matplotlib \cite{matplotlib}, to generate the plots and visualizations.

VS Code was used as an IDE for code implementation, and data analysis of the results was done using Jupyter Notebook \cite{jupyter}.

The server used for training and evaluation of the neural network model has the hardware specifications described in Table \ref{tab:serverHW} below.

\begin{table}[h]
    \caption{Server Hardware specifications}
    \centering
    \begin{tabular}{|c|l|}
    \hline
    \multirow{3}{*}{CPU} & Intel(R) Xeon(R) Gold 5118 CPU       \\
                         & 2.30GHz of frequency                 \\
                         & 12 cores                             \\ \hline
    RAM                  & 192 GB                               \\ \hline
    \multirow{3}{*}{GPU} & 2x Nvidia Tesla V100 with 16 GB or RAM each \\
                         & Nvidia Volta Microarchitecture       \\
                         & Compute Capability 7.0               \\ \hline
    \end{tabular}
    \label{tab:serverHW}
\end{table}



% ============================================================
%
%                   CHAPTER 5: RESULTS
%
% =============================================================
\input{5_Results.tex}

% ============================================================
%
%                   CHAPTER 6: CONCLUSION
%
% =============================================================

\chapter{Conclusion}
\label{ch:Conclusion}

% % --------------------------------------
\section{Takeaways}
\label{sec:Takeaways}
% % --------------------------------------
The following are the key takeaways of this project:
\begin{enumerate}[label=\alph*.]
\item Apple Neural Engine (ANE) Utilization: This project explored leveraging the underused NPU for LLM inference, going beyond standard CoreML use cases and showed that it can be viable.

\item Retrieval-Augmented Generation (RAG) without Internet : Implements a lightweight RAG pipeline that only uses on device resources.

\item Portability and Ease of use: Application size of less than 1GB that includes everything, available for download at \href{https://tldr.cool}{\textbf{https://tldr.cool}}

\item Quantized LLMs: Uses compact models (50–500MB) for efficient, on-device inference allowing for seamless multitasking and manages to obtain results comparable to mainstream cloud LLMs.
\end{enumerate}
% % --------------------------------------
\section{Limitations}
\label{sec:Limitations}
% % --------------------------------------
Following are the main limitations uncovered during the implementation of this project:
\begin{enumerate}[label=\alph*.]
\item While this project demonstrates the feasibility to accelerate the retrieval part of the RAG pipeline, the LLM Inference still only leverages the GPU and does not leverage the NPU.

\item Indexing(embedding) takes 90+% of overall runtime

\item Many smaller LLMs available currently (3B or less) are only instruction-tuned i.e trained for text completion and not for chat. This can lead to unfavorable responses during chat.

\item Excessive usage limits to quick draining of power, especially on portable devices.

\item The breadth of retrieval space is limited not by resources but by the context size of the Chat LLM, since both input and expected output must fit in the LLM context.

\end{enumerate}
% % --------------------------------------
\section{Future work}
\label{sec:FutureWork}
% % --------------------------------------
\begin{enumerate}[label=\alph*.]
    \item Enhance data safety mechanisms for vectordump files.
    \item Add NPU backend for GGML and llama.cpp.
    \item Quantize and convert fine-tuned chat-optimized LLMs to the GGUF format.
    \item Implement a dedicated parallelized tokenization module.
    \item Extend the application support beyond Apple Silicon.
    \item Add NPU and GPU acceleration support for SQLite and Postgres vector search extensions (they currently only support CPU).
    \item Create an optimized decoding and tokenization workflow dedicated for embedding (e.g., no KV cache).
\end{enumerate}

\printendnotes

%
% ==========   Bibliography
%
% \nocite{*}   % include everything in the uwthesis.bib file
\bibliographystyle{plain}
\bibliography{uwthesis}
%
% ==========   Appendices
%

\end{document}
